\chapter*{Conclusion}
\addcontentsline{toc}{chapter}{Conclusion} 

The thesis describe my personal contribution in the context of the IDEA Experiment, in particular of the dual-readout calorimeter software.\\

Therefore, in the fist part the main goals in modern particle phisycs have been described, with a further illustration of the main future accelerators. The IDEA detector concept has been outlined in all its parts, being one of the multi-detector that could take place in the leptonic accelerators.\\

The second chapter detailed the theoretical background on calorimetry, from the basic concepts to more modern techniques. The focus has been placed on the dual readout compensation technique, cornerstone of the IDEA calorimeter, exploiting its strengths and weaknesses.\\

Then, a whole chapter has been dedicated to Silicon PhotoMultipliers, photo-detectors that represent an alternative to PhotoMultiplier Tubes. The working principle of these devices are described along with their characterization features as gain, photon detection efficiency, linearity and noise effects.\\

The last two chapters have shown my software contribution to the  IDEA dual-readout calorimeter. It started from the study of the simulation chain (modelling calorimeter and SiPM) to evaluate the performances and validate the full simulation. Finally, the development of deep learning algorithms has been shown, with the achievement of excellent results in performing particle identification distinguishing the 2D spatial distribution of released energy by neutral pions and photons.