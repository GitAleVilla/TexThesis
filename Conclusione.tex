\chapter*{Conclusion}
\addcontentsline{toc}{chapter}{Conclusion} 

The thesis describes my personal contribution in the context of the IDEA experiment for future $e^+ e^-$ electroweak factories and its dual-readout calorimeter simulation.\\

The main goals in modern particle physics have been described focusing on the studies on the Higgs boson. A picture of the main future colliders has been provided, pointing out their capabilities and limitations. The IDEA detector concept has been outlined in all its parts, being one of the multi-purpose detector proposed for future $e^+ e^-$ electroweak factories.\\


The second chapter has detailed the theoretical background on calorimetry, from the basic concepts to more modern techniques. The focus has been placed on the dual-readout compensation technique, cornerstone of the IDEA calorimeter, explaining its strengths and weaknesses.\\

The third chapter has been dedicated to Silicon PhotoMultipliers (SiPMs), photo-detectors that represent today's alternative to PhotoMultiplier Tubes for the IDEA dual-readout calorimeter. The working principle of these devices has been described along with their characterisation features as gain, photon detection efficiency, linearity and noise effects.\\

The last two chapters has described my contributions to the IDEA dual-readout calorimeter software. It has begun with the study of the simulation chain (modelling both the calorimeter response and the SiPM transfer function) to evaluate the performance and validate the full simulation.\\
In this context, I have shown that the impact of the SiPM digitisation on the temporal distributions behaves as expected.
The readout linearity has been studied, showing that a $1\%$ linear detector for electromagnetic shower detection, considering both the calorimeter and the readout system, can be achieved, at least in the energy range studied ($20-80$ GeV).\\
%, applying an analytical correction.\\

Eventually, the development of deep learning algorithms has been shown, together with the achievement of excellent results in performing particle identification tasks, like distinguishing between neutral pions and photons on the basis of their spatial shower distribution.
Two neural network (NN) structures were considered, a VGG-like NN and a ResNet-like NN. I have shown that they can achieve excellent performance in identifying $\pi^0$'s with accuracy values of $98.9\%$ (VGGNet) and $97.6\%$ (ResNet) while rejecting $\gamma$'s with probabilities of $99.4\%$ (VGGNet) and $98.3\%$ (ResNet).
%I showed that a VGG Network and a Residual Network set up in the simplified problem with a fixed particle energy of $40$ GeV have an accuracy over $98\%$ and $97\%$ respectively.\\
%The final results, where the two Neural Networks have been applied on events generated by particles of energy ranging from $1$ up to $80$ GeV, can be summarised reporting the accuracy of $99.15\%$ (VGGNet) and $97.95\%$ (ResNet) and the AUC of $0.9982$ (VGGNet) and $0.9957$ (ResNet). 
These results demonstrate the great potential of deep neural network methods applied on data from the IDEA dual-readout calorimeter. The high granularity of the detector and the capability of deep neural network to effectively handle huge amount of data may provide the key for a significant step forward for future calorimetry at high-energy $e^+ e^-$ colliders.