\documentclass[a4paper,11pt,titlepage]{book}	%classe e struttura
\usepackage[italian]{babel} 	%lingua principale
\usepackage[utf8]{inputenc}	%caratteri accentati
\usepackage{graphicx}		%pacchetto per le immagini
\usepackage{mathrsfs}		%pacchetto per scrittura manoscritto
\usepackage{booktabs}		%pacchetto per separatori tabelle
\usepackage{multirow}		%pacchetto per multirighe in tabelle
\usepackage{siunitx}		%pacchetto per le tabelle dati
\usepackage{wrapfig}		%pacchetto per le immagini immerse nel testo
\usepackage{latexsym}		%pacchetto per lettere greche
\usepackage{tabularx}		%pacchetto per tabelle 
\usepackage{amsmath}		%pacchetto per matematica
\usepackage{physics}		%pacchetto per fisica bra ket
\usepackage{subfig}		%pacchetto per sottotitoli a immagini
\usepackage{tikz}		%pacchetto per disegni
%\usepackage{circuitikz}	%pacchetto per disegni di circuiti
\usepackage{fancyhdr}
\usepackage{gensymb}
\usepackage{url}

\usepackage{frontespizio}

\frenchspacing 		%regola spazi
\raggedbottom		%spazio vuoto a fine pagina, quando finisce il testo, se lo togli allunga gli spazi per riempire nell'altezza la pagina

%\title{Fotorivelatori Criogenici per rivelazione di eventi rari in fisica delle alte energie}
%\author{Villa Alessandro}
%\date{27 febbraio 2018}

%\pagestyle{fancy}\addtolength{\headwidth}{20pt}\addtolength{\headheight}{20pt}
%\fancypagestyle{plain}{\fancyfoot[RO,LE]{\normalsize\thepage}} %%% numeri prima pagina

%\makeatletter
%\def\cleardoublepage{\clearpage\if@twoside \ifodd\c@page\else
%\hbox{}
%Potresti voler togliere il commento dalla linea seguente
%Questa pagina è stata lasciata intenzionalmente vuota.
%\thispagestyle{empty}
%\newpage
%\if@twocolumn\hbox{}\newpage\fi\fi\fi}
%\makeatother

\begin{document}
	
	%---------FRONTESPIZIO-----
	\begin{frontespizio}
		\Universita{Pavia}
		\Logo[3cm]{logo}
		\Facolta{scienze matematiche, fisiche, naturali}
		\Corso[laurea]{Scienze Fisiche}
		\Annoaccademico{2019-2020}
		%\Titoletto{Tesi di laurea magistrale}
		\Titolo{Fotorivelatori Criogenici per la rivelazione di eventi rari\\ in fisica delle alte energie}
		\Punteggiatura{}
		\Candidato[462495]{Alessandro Villa}
		\NRelatore{Supervisore}{Supervisori}
		\Relatore{Dott. Andrea Negri}
		\NCorrelatore{Co-Supervisore}{Co-Supervisori}
		\Correlatore{Dott. Roberto Ferrari}
		\Correlatore{Dott. Lorenzo Pezzotti}
	\end{frontespizio}
	
	%_________________________CORPO INIZIALE
	\begin{frontmatter}
		\tableofcontents
		
		%---------INTRODUZIONE------
		\chapter{Introduction}
The discovery of the Higgs boson in 2012 at CERN has opened the door for several new studies in particle physics.
%Its relatively small mass allows to produce it in high luminosity electron-positron colliders.
The Higgs boson plays a crucial role in the biggest mysteries of modern particle physics and the precise measurement of its properties, together with the properties of the $W$ and $Z$ bosons, will provide deep information about the Standard Model (SM) and the physics Beyond it (BSM). At present, several projects are ongoing to design the next big-collider and its detectors with the primary task of measuring the Higgs properties.\\

In the first chapter, a short introduction to the main goals of modern particle physics is presented. The current design of possible future electroweak collider facilities is described, pointing out their capabilities to improve our knowledge of the field.
Finally, the IDEA detector concept is outlined, being one of the multi-purpose detector concepts at future electroweak colliders.\\

The second chapter provides a theoretical background on calorimetry, starting from the interaction between particles and matter and the particle showering mechanism. A description of modern calorimeters is also provided, distinguishing between electromagnetic and hadronic calorimeters. Eventually, the dual-readout compensation technique, cornerstone of the IDEA calorimeter, is introduced.\\

The third chapter is dedicated to the Silicon PhotoMultipliers. They are the photodetectors envisaged to be coupled to the IDEA dual-readout calorimeter. Their working principle is described, followed by their characteristic features such as gain, photon detection efficiency, linearity and noise sources.\\

The fourth chapter describes the simulation chain of the IDEA dual-readout calorimeter combining the Monte Carlo simulation of the calorimeter with the readout system simulation, emulating the SiPM response. Several analyses were performed to evaluate the performance and validate the full simulation model. They are described here for the first time.\\

The last chapter introduces the application of deep learning algorithms to the IDEA calorimeter simulated data. Two different algorithms have been used to perform the very challenging task of discriminating between neutral pions and photons. Indeed, the high spatial resolution of the IDEA calorimeter allows to reconstruct the two-dimensional pattern of particle showers with great accuracy, thus providing a powerful input for neural-network applications. Encouraging results were found and are extensively reported in the thesis.\\
		
	\end{frontmatter}
	
	%_________________________CORPO CENTRALE
	\begin{mainmatter}
		%---------CAPITOLO 1--------
		\chapter{Future colliders}
The chapter starts with a short description of the most important goals in the particle physics framework starting from the possibilities opened with the discovery of the Higgs boson.\\
It continues, in section \ref{sec:Coll-ee}, with an overview on the different projects of leptonic colliders with circular or linear structure.\\
At the end, the IDEA Detector concept is described. It is a multidetector project included in the conceptual design reports from both the two circular leptonic collider collaborations.

\section{Physics goals}
With the discovery of the Higgs boson ($H$) in 2012 by ATLAS and CMS Collaborations \cite{ATLAS_H, CMS_H}, a new era has been opened where the boson will not only be the object of researches but also the tool for new particle physics studies.\\

The total Higgs production cross section can be measured in the Higgstrahlung process ($e^+ e^- \rightarrow Z H$) looking at its presence as the recoil to the $Z$. In this way the $H$ boson is directly coupled to the $Z$ measurement.
The $H Z$ events have their recoil mass equal to $m_H$, hence Higgs bosons can be counted from the accumulation around the $H$ mass value, allowing to determine the $H Z$ cross section ($\sigma_{HZ}$).
In particular, the total cross section could be determined independently of the $H$ decay, by counting the Higgsstrahlung events characterised by a leptonic decay: $Z \rightarrow l^+ l^-$.
This method disentangled the $H$ production from its decay, providing a model-independent determination of its total width and its other couplings through branching ratio measurements with a sub-percent precision.\\

The $H$ couplings to the first Standard Model family particles (i.e. electron, quark up and quark down), because of their small masses and related decay branching ratios, will not be directly measurable at these colliders. 
However, with beam energies $\simeq 125.09\ GeV$, corresponding to the $H$ pole mass, leptonic colliders can contribute to set upper limits to the electron Yukawa coupling taking advantage from the resonant $H$ production.
Also the $t$ Yukawa and the $H$ self-couplings will not be directly measurable because their masses are too large for a kinematically open decay.
The only leptonic collider operating at $\sqrt{s} = 350\ GeV$ (FCC-ee) could exploit the $t\Bar{t}$ cross section accessing to the $t$ Yukawa with a precision of $\simeq 10\%$.\\

It is possible to predict mass and properties of the top-quark and of the bosons $Z$, $W$ and $H$ using the Standard Model. This predictions are obtained from precise measurements in the Electroweak sector and theoretical calculations.
Quantities called Electroweak Precision Observables (EWPOs) can establish the presence of new physics. For this reason EWPOs measurements represent another important component in the future colliders physics program.
The future leptonic colliders has the purpose to significantly improve in the precision (more than an order of magnitude actual one) their value and provide a broad set of EWPOs, giving access to many possible new physics sources.
Therefore, the main  studies will concern the $Z$ pole, the light neutrino species number and the $W^+W^-$ and $t\Bar{t}$ thresholds.

\section{Leptonic colliders}\label{sec:Coll-ee}

The Higgs boson can be produced in new $e^+e^-$ colliders, with high luminosity thanks to its relatively small mass.
Precise measurements of its properties, together with those of the $Z$ and $W$ bosons, will provide important tests of the SM fundamental physics principles and will be essential for the physics Beyond the Standard Model (BSM) studies.
A roadmap to achieve the results previously described is already set (see figure \ref{fig:roadmap}).
At present, the program includes 4 different leptonic collider:
\begin{itemize}
    \item the \textit{Future Circular Collider} (FCC-ee), at CERN;
    \item the \textit{Circular Electron Positron Collider} (CEPC), in China;
    \item the \textit{International Linear Collider} (ILC), in Japan;
    \item the \textit{Compact LInear Collider} (CLIC), at CERN.
\end{itemize}
In the following sections, a brief description of each one of them will be provided.

\begin{sidewaysfigure}
	\includegraphics[width = 0.95\textwidth]{IMG/Cap1/Roadmap.png}
	\caption{Possible  timelines  of  future  colliders.  It includes $e^+e^-$ (ILC,  CLIC,  CEPC  and  FCC-ee), $pp$ (FCC-hh  and HE-LHC) and $e-p$ (LHeC and FCC-eh) machines. Image from \cite{roadmap}.}
	\label{fig:roadmap}
\end{sidewaysfigure}

\subsection*{Future Circular Collider}
A post-LHC circular collider has been proposed by the CERN with the name of Future Circular Collider (FCC) project \cite{FCC}. FCC is staged in a first lepton collider (FCC-ee) \cite{FCC-ee} phase followed by a hadron collider (FCC-hh) \cite{FCC-hh}. A common tunnel about $100\ km$ long is designed to host both of them. With this choice the same facility could also house an electron-hadron collider.\\
FCC-ee is designed to provide the highest possible statistics for the $Z$, $W$ and $H$ bosons, and $t$ quarks. For this reason FCC-ee will operate at center-of-mass energies ranging from $88$ to $365\ GeV$ in four different $\sqrt{s}$ operating points:
\begin{itemize}
    \item $\simeq 91\ GeV$, corresponding to the $Z$ pole;
    \item $\simeq 160\ GeV$, corresponding to the $W^+W^-$ production threshold;
    \item $\simeq 240\ GeV$, corresponding to the $ZH$ production threshold;
    \item $\simeq 340-365\ GeV$, corresponding to the $t\Bar{t}$ threshold.
\end{itemize}

The FCC-ee project integrates with the present CERN accelerator complex, where the injector chain makes use of a $6\ GeV$ linac, a damping ring and the CERN SPS as a pre-booster. The layout, sketched in figure \ref{fig:FCC-ee}, is designed with two different interaction points.\\
The task of increasing our knowledge of electroweak and Higgs observables by one or two orders of magnitude better than the current one sets challenging constrains to the luminosities needed. These values range from $0.2\ ab^{-1}$, for the measurement of the top-quark mass and width, to $100\ ab^{-1}$, for the measurement of the effective weak mixing angle.
These luminosities can be achieve in a reasonable amount of time only circular colliders. 

\begin{figure}
	\centering
	\includegraphics[width=.7\textwidth]{IMG/Cap1/FCC-ee.png}
	\caption{ Schematic view of the FCC-ee. Image from \cite{FCC}.}
	\label{fig:FCC-ee}
\end{figure}

\subsection*{Circular Electron Positron Collider}
The Circular Electron Positron Collider (CEPC) is an international project initiated and hosted by China. It is designed to operate at centre-of-mass energies of $240\ GeV$ (as $H$ factory via $e^+e^- \rightarrow ZH$), $91.2\ GeV$ ($Z$ pole) and 160 GeV ($W^+W^-$ threshold scan).\\

The collider is composed by a double ring, sketched in figure \ref{fig:CEPC} with a circumference of $100\ km$ and two interaction points. The same $100\ km$-long tunnel could also host a Super Proton-Proton Collider (SPPC), also without removing the CEPC giving the possibility of a electron-proton collision. As described in the conceptual design report \cite{CEPC_design1, CEPC_design2}, the main accelerator is preceded by a linear accelerator, a damping ring and a booster.\\
Associated to the three operating $\sqrt{s}$ values the instantaneous luminosities are expected to reach $3 \times 10^{34}$, $32 \times 10^{34}$ and $10 \times 10^{34}\ cm^{-2}s^{-1}$, respectively.
CEPC will produce, over is planned operative time, large samples (more than one million) of Higgs, one trillion of $Z$ bosons and about 100 million of $W^+W^-$ events, allowing precision measurements of their properties.\\

\begin{figure}
	\centering
	\includegraphics[width=.7\textwidth]{IMG/Cap1/CEPC.png}
	\caption{ Schematic view of the CEPC. Image from \cite{CEPC_design2}.}
	\label{fig:CEPC}
\end{figure}

According to \cite{CEPC_schedule}, the CEPC construction could be possibly start within few years and could be completed by 2030, by far the most aggressive time schedule among the big collider proposals.

\subsection*{International Linear Collider}
The ILC is an $e^+e^-$ linear collider that, at the current state (ILC250), provides a $250\ GeV$ center-of-mass energy. This value is extendible to $1\ TeV$.
The first stage of ILC has the task to measure the $H$ parameters and their model-independent determination studying the $e^+e^- \rightarrow ZH$ collision at $\sqrt{s}= 250\ GeV$.
Two energy upgrades are currently projected extending the center-of-mass energy to $500\ GeV$ and $1\ TeV$.
The higher-energy physics goals are to increase the precision on the measurement of the top-quark mass, of the top-quark electroweak couplings, of the Higgs coupling to the top quark, and of the triple-Higgs coupling \cite{ILC_global_project}.\\

It will also search for new physics hints in exotic decays of $H$ and in pair-production of weakly interacting particles. A sketch of the ILC250 is shown in \ref{fig:ILC250}.\\
The ILC accelerator is based on Super Conducting Radio Frequency (SCRF) cavities already in use in the European X-ray Free Electron Laser (E-XFEL), at DESY/Hamburg.
The estimated power consumption at the three centre-of-mass energy stages of the baseline option are \cite{ILC_design}: $122\ MW$ (at $250\ GeV$), $121\ MW$ (at $350\ GeV$) and $163\ MW$ (at $500\ GeV$). It increases to $300\ MW$ for the $1\ TeV$ option.\\
In its current state, the SCRF cavities will reach frequency values of $1.3\ GHz$ providing a gradient of $31.5-35\ MV/m$ while operating in a cryogenic infrastructure at $2\ K$.\\
The design luminosity is $1.35-1.5 \times 10^{34}\ cm^{-2}s^{-1}$, with a corresponding integrated luminosity from $400\ fb^{-1}$ (in the first years) to $2\ ab^{-1}$ (after future upgrades).
The ILC250 is $20.5\ km$ long with two main arms, mostly occupied by the electron and positron linacs, at $14\ mrad$ crossing angle. The ILC candidate site is in the Kitakami region in northern Japan.

\begin{figure}
	\centering
	\includegraphics[width=.8\textwidth]{IMG/Cap1/ILC.png}
	\caption{Schematic layout of the ILC at 250 GeV staged option \cite{ILC_global_project}.}
	\label{fig:ILC250}
\end{figure}

\subsection*{Compact Linear Collider}
The Compact Linear Collider (CLIC) is a TeV-scale high-luminosity linear leptonic collider to be located in the CERN area. The CLIC energy stages are $\sqrt{s}= 380\ GeV$, $1.5\ TeV$ and $3.0\ TeV$, with corresponding instantaneous luminosities of $1.5$, $3.7$ and $5.9 \times 10^{34}\ cm^{-2}s^{-1}$, respectively. The site length scales, in these stages, from $11\ km$ up to $50\ km$.\\

The CLIC project, proposed in 2012 \cite{CLIC_old1, CLIC_old2, CLIC_old3} updated in 2016 \cite{CLIC_update}, will adopt a two-beam acceleration scheme as shown in figure \ref{fig:CLIC} where electrons and positrons beams are independent through the whole chain.\\
In the first stages, particles are accelerated to $9 GeV$ using a linac booster. Then they are injected in normal-conducting high-gradient $12\ GHz$ accelerating structures. The two main linacs accelerate beams exploiting normal conducting $X$-band cavities with an accelerating gradient of $100\ MV/m$. To reach these extremely high accelerating gradients a novel drive-beam scheme that uses low-frequency klystrons to generate long RF pulses and stores their energy in a long, high-current, drive-beam pulse. This beam pulse is used to generate several short pulses distributed along the main linac.

\begin{figure}
	\centering
	\includegraphics[width=.8\textwidth]{IMG/Cap1/CLIC.png}
	\caption{Schematic layout of the CLIC at 380 GeV staged option \cite{CLIC_img}.}
	\label{fig:CLIC}
\end{figure}

\section{IDEA Detector concept} \label{sec:Idea_project}
IDEA (Innovative Detector for an Electron-positron Accelerator) is an innovative general-purpose detector concept, designed to study electron-positron collisions in a wide energy range provided by a very large circular leptonic collider, with a typical circumference of 100 km.
The new detector concept called was proposed in 2017 and was included in conceptual design reports of both FCC-ee \cite{FCC-ee_design} and CEPC \cite{CEPC_design}.

\begin{figure}
	\centering
	\subfloat[][ Artistic view of the IDEA detector concept. \label{fig:IDEA_art1}]{\includegraphics[width=.45\textwidth]{IMG/Cap1/IDEA_art1}} \quad
	\subfloat[][The structure and dimensions of the IDEA detector concept. \label{fig:IDEA_art2}]{\includegraphics[width=.45\textwidth]{IMG/Cap1/IDEA_art2}}
	\caption{IDEA Detector concept.}
\end{figure}

The IDEA Detector concept is sketched in an artistic view in figure \ref{fig:IDEA_art1} and in its structure and dimensions in figure \ref{fig:IDEA_art2}. 
The overall detector is composed by a silicon pixel vertex detector, a wire chamber surrounded by a layer of silicon micro-strip detectors, a thin superconducting solenoid coil, a pre-shower detector, a dual-readout calorimeter, and muon chambers.
The most innovative elements proposed with this project are:
\begin{itemize}
  \item the ultra-light drift chamber as main tracker;
  \item the Dual-Readout (DR) fiber calorimeter.
\end{itemize}

The drift chamber technology is based on an upgrade of MEG, designed to search for the charged lepton flavor violating decay $\mu \rightarrow e^+\gamma$. The R\&D work studies the construction and operation of the MEG2 Ultra Light TIming Drift Chamber (MULTIDC) performing high momentum resolution and transparency in terms of radiation length.
The IDEA dual-readout calorimeter, on the other hand, stands on the legacy of the DREAM/RD52 Collaboration. The key point is the feasibility of dual-readout optical-fibers calorimeters to obtain high resolution in energy measurement associated to both single-hadron and hadronic jets.
All the most important parameters of the IDEA detector components are listed in table \ref{tab:IDEA_part}.\\
\begin{table}
  \centering
  \begin{tabular}{ll}
    \toprule
    Vertex technology                       & Silicon \\
    Vertex inner/outer radius (cm)          & $1.7/34$ \\
    \midrule
    Tracker technology                      & Drift Chamber and Silicon Wrapper\\
    Tracker half length (m)                 & $2.0$ \\
    Tracker outer radius (m)                & $2.0$ \\
    \midrule
    Solenoid field (T)                      & $2.0$ \\
    Solenoid bore radius / half length (m)  & $2.1/30.$ \\
    \midrule
    Preshower absorber                      & Lead \\
    Preshower $R_{min}/R_{max}$ (m)         & $2.4/2.5$ \\
    \midrule
    Calorimeter absorber                    & Copper \\
    Calorimeter $R_{min}/R_{max}$ (m)       & $2.5/4.5$ \\
    \midrule
    Overall height / length (m)             & 11/13 \\
    \bottomrule
  \end{tabular}
  \caption{Parameters of the different sub-detectors composing IDEA.}
  \label{tab:IDEA_part}
\end{table}

\subsection{Vertex detector}
The $1.5\ cm$ beam pipe is surrounded by the IDEA vertex detector composed by pixel active sensors. The structure present high-resistivity substrates architectures implementing on-pixel sparsification and data driven,time stamped readout.
The goal is a thickness of $0.15-0.30\% X_0$ per layer and a power dissipation below $20\ mW/cm^2$.
The vertex detector measures tracks of charged particles with very high precision, of the order of $3\ \mu m$ in the innermost layers, and is able to precisely reconstruct secondary vertices.
This detector will significantly benefit from the electronic and mechanical work for the ALICE ITS \cite{alice_its}, as well as of new ongoing developments, in the framework of the INFN ARCADIA R\&D project.

\subsection{Drift chamber}
The IDEA drift chamber project consist in an ultra-light Drift CHamber (DCH). Following the results obtained by the KLOE Experiment DCH \cite{KLOE} and the recent DCH for MEG2 (the MEG upgrade) \cite{MEG2}, a detector with extraordinary transparency to radiation is designed.\\

The chamber is composed by a unique cylindrical volume, co-axial to the beam axis, with an inner radius of $0.35\ m$ and an outer radius of $2\ m$, for a total length of $4\ m$. It consists of $112$ co-axial layers, at alternating sign stereo angles, grouped in $24$ identical sectors. The ammount of drift cells is $56448$ with variable size from $12.0$ to $14.5\ mm$.
The chamber is operated with a very light gas mixture of $90\%$ He - $10\%$ iC$_4$H$_{10}$ (isobutane), providing a maximum drift value of $\simeq 400\ ns$.\\
The angular coverage extends down to $\simeq 13\deg$, and could be extended with additional silicon disks between the DCH and the calorimeter end caps.\\
In the radial direction the total amount of material is of the order of $1.6\%$ of a radiation length, including inner and outer cylindrical walls and contributions from the gas mixture and the wires. On the other hand, in the forward and backward direction, the total amount of material is equivalent to about $5.0\% X_0$, including inner cylindrical walls and services end plates, instrumented with front-end electronics, signal and HV cables.\\

In the context of the MEG2 drift chamber prototypes \cite{MEG2} with $7\ mm$ cell size, a drift distance resolution of $100\ \mu m$ has been achieved with both a gas mixture and an electrostatic conditions very similar to the ones foreseen for the IDEA DCH.
Analytical calculations assuming this resolution for the DCH allow to study the momentum and angular resolution with the result shown in figure \ref{fig:DCH_res}.
The drift chamber also offers outstanding particle-identification performance using the cluster counting technology improving both spatial resolution and particle identification.

Together with the excellent expected momentum resolution, the DCH can achieve superior particle identification capabilities thanks to the cluster-counting technique. The ionization process for which electrons are released follows a Poison law, therefore by counting the total number of ionization clusters ($N_{cl}$) of a charged track one can reach a relative resolution on $N_{cl}$ that follows $1/\sqrt{N_{cl}}$. The expected performance relative to particle separation in terms of number of standard deviations as a function of the particle momentum are shown in figure \ref{fig:DCH_separation}. In this graph the solid curves refer to the cluster-counting technique, while the dashed one refers to the expected identification power for the traditional $dE/dx$ method. As it can seen, the particle separation by cluster counting is more performing in the whole range of momentum.

\begin{figure}
	\centering
	\subfloat[][Momentum and angular resolutions for $\theta= 90 \deg$ as a function of momentum. \label{fig:DCH_res}]{\includegraphics[width=.45\textwidth]{IMG/Cap1/DCH_res.png}} \quad
	\subfloat[][Particle type separation in units of standard deviations as a function of momentum, with cluster counting (solid curves) and with $dE/dx$ (dashed curves). \label{fig:DCH_separation}]{\includegraphics[width=.45\textwidth]{IMG/Cap1/DCH_separation.png}}
	\caption{IDEA ultra-light drift chamber performances \cite{FCC-ee_design}}
\end{figure}

\subsection{Magnet system}
The IDEA detector magnet is an ultra-thin and light (thus “radiation-transparent”) superconducting solenoid. It is $5\ m$ long and has an inner diameter of $4.2\ m$. equipped with a thin iron return yoke. The main feature is that the solenoid is positioned between the tracker block and the calorimeter, a solution currently employed in ATLAS.
This choice impose to keep the total thickness at $30\ cm$ level, below $1 X_0$ in terms of radiation length, but at the same time the stored energy is reduced by a factor four and the cost can be halved.
In this scenario a relatively low field of $2\ T$ can be produced.
The flux return yoke scales with the square of the coil diameter, thus with the given dimensions a yoke thickness of less than $100\ cm$ of iron is sufficient to contain the magnetic flux and to shield the muon chambers.

\subsection{Dual-readout calorimeter}
The IDEA calorimeter consist in a dual-readout projective fiber detector. It is designed starting from the studies on dual-readout performed in calorimeters as SPACAL \cite{SPACAL} and RD52 \cite{RD52}.\\
The detector is a $4\pi$ calorimeter that does not present segmentation longitudinally, but only in the direction towards the Interaction Point (IP).
The segmentation is chosen subtle to have the shower development confined in a small number of cells and most of the energy deposited in a single cell.
This highly simplifies the calibration procedures for which each cell response can be considered individually.
Considering that in accelerator-based experiments all the particle come, in good approximation, from the IP, the projective segmentation can be obtained with a tower-based structure.\\

DR calorimeters, as described in section \ref{sec:DRComp}, are composed by an absorber and two different active media to induce and transport two different signals. In the IDEA DR calorimeter copper is used as absorber, composing the structure of the towers, and filled with optical fibres as active volumes.
The possibility to independently readout each fiber with a dedicated Silicon PhotoMultiplier (SiPM), as described in \cite{Massi_tesi}, brings a series of advantages especially in terms of spatial and angular resolution removing the limitations given by the towers. Also two particle showering in the same tower, in this way, can still be identified.\\

\begin{figure}
	\centering
	\includegraphics[width=.9\textwidth]{IMG/Cap1/DRCal_geo.png}
	\caption{IDEA dual-readout calorimetry geometry produced with GEANT4 from different angles. In orange the towers composing the barrel, in blue the ones composing the endcaps.}
	\label{fig:DRCal_geo}
\end{figure}

The geometry of the calorimeter is shown from different angles in figure \ref{fig:DRCal_geo}. To have projective towers the solution proposed is a series of truncated pyramids pointing to the IP. In such a way, each tower cover a specific region of the solid angle.\\
The cylindrical symmetry is obtained producing a rotation around the beam axis of a minimal structure called slice. A single slice cover a range of $1o\degree$ of the $\varphi$ angle, therefore $36$ of these elements cover all the calorimeter volume.
In each slice both barrel and endcaps towers are present, in particular $80$ towers for the barrel and $35$ for each endcap, with a $\theta$ coverage of $1.125\degree$ for each tower.\\
All the towers are $2\ m$ long, composed by copper and arranged to be filled by optical fibres.
The active elements are scintillating (polystyrene) and clear-plastic fibers (PolyMethyl MethAcrylate - PMMA).
The fiber diameter is $1\ mm$ thick (core + cladding) and disposed in a chess-board like geometry so that each fiber is separated from the closest ones by $0.5\ mm$ of absorber material.
This complex geometry has been reproduced within the GEANT4 simulation toolkit \cite{GEANT4} and all results in this thesis are obtained with it.

\subsection{Preshower and muon chambers}
Both the IDEA preshower and the muon chambers are based on the micro-Resistive WELL($\mu$-RWELL) technology [35]. $\mu$-RWELL chambers are compact Micro-Patter Gaseous Detector (MPGD), with a single amplification stage intrinsically spark protected.\\

A preshower detector is located in the barrel region between the magnet and the calorimeter and another one in the forward region between the drift chamber and the end-cap calorimeter.
In the barrel region, the magnet coil play also the role of of about $1 X_0$ and it is followed by one layer of MPGD chambers; a second layer of chambers follows after another $1 X_0$ of lead. In the forward region an analogue structure is built with both the $\mu$-RWELL layers preceded by $1 X_0$-long lead absorber.\\
The evaluation of the preshower performance and the single-hit-position resolution requirement are still in progress.
But with the now-a-day performances already allow to tag $\simeq 30\%$ of the $\pi^0$ from their $\gamma\gamma$ decay and provides good acceptance for photons.
This technology will perfectly match also the requirements for the IDEA muon system, providing a good tracking efficiency, high-voltage stability, a space resolution for the coordinates of a muon track of $200-300\ \mu m$ and a good time resolution thank to the fast charge amplification process.\\

Also the muon detector uses as well the $\mu$-RWELL technology but with a wider strip pitch, due to the greater dimensions. It is subdivided in three active layers at increasing distance from the vertex, and located within the iron return yoke that closes the magnetic field. Each MPGD can provide a space point with a spatial resolution of about $400\ \mu m$ in the plane perpendicular to the particle direction. Combining the three stations allows to perform standalone tracking of charged particles at $5-6\ m$ from the vertex. Such a precision also allows to identify secondary vertices that could be produced by long lived particles.
		
		%---------CAPITOLO 2--------
		\chapter[Calorimetry and the DR method]{Calorimetry and the dual-readout method}
%Calorimetry is an important detection principle in particle physics.
Originally developed with astrophysical purpose for cosmic-ray studies, the art of calorimetry refers to the detection of particles and the measurement of their properties by total or partial absorption using blocks of instrumented material.
It was developed and perfected for accelerator-based particle physics experimentation primarily in order to measure the energy of particles. 
In calorimeters, particles are fully absorbed and their energy transformed into a measurable signal.\\
The incident particle interact with the detector (through electromagnetic or nuclear processes) producing a shower of secondary particles with progressively degraded energy.
The energy deposited by charged particles in the calorimeter active material is used to generate signals, typically in the form of charge or light.
Two typical processes exploited are the scintillation light emitted in response of ionization, or the Cherenkov light produced by relativistic particles.\\

Calorimeters can be divided into two categories depending on the type of shower they are optimized to detect: electromagnetic calorimeters, used mainly to measure electrons and photons energies through their electromagnetic interactions with the detector material, and hadronic calorimeters, used to measure hadrons energies through their strong and electromagnetic interactions.
Another classification can be made according to their structure dividing between sampling calorimeters and homogeneous calorimeters.\\
Homogeneous calorimeters are built of one type of material that performs two tasks: it degrades the energy of the incident particles and provides the detectable signal.
Sampling calorimeters, instead, consist of alternating layers of an absorber, a dense material used to absorb particle showers, and an active medium that generate the signal.\\

This chapter describes the physics behind both the electromagnetic and hadronic shower developments, provides a basic description of the energy response of these detectors and introduces the dual-readout calorimetry technique.\\

A more comprehensive descriptions of the field can be found in \cite{Wigmans_book, Wigmans_art_of_cal, Gianotti_article}.

\section{Physics of shower development}
The groundwork for the calorimetry is the interaction processes happening between particles and matter.
These processes depend on the absorbing medium, the particle type and energy.
While absorbing an energetic particle, a cascade of subsequent particles is formed known as a particle shower.
The processes involved and the calorimeter response to the showering particles are the keys to deeply understand this topic.\\

\subsection{Electromagnetic showers} \label{subsec:em_shower}
Electromagnetic showers are governed by a small number of well understood QED processes. Charged particles (electrons and positrons) lose energy by ionization and by radiation, instead neutral ones (photons) interact with matter by photoelectric effect, Compton scattering and pair production.\\

Therefore, electrons and positrons ionize the medium under the condition of having an energy at least sufficient to release the atomic electrons from the Coulomb fields generated by the atomic nuclei (few eV).
The amount of energy released (in unit of path length) is predictable through the semi-empirical Bethe-Block formula restricted to electrons (and positrons) \cite{Leo}:
\begin{equation}
    -\frac{dE}{dx} = 2\pi N_a r_e^2 m_e c^2 \rho \frac{Z}{A}\frac{1}{\beta^2}\left[ \ln{\frac{\tau^2(\tau + 2)}{2(I/m_ec^2)^2}} -F(\tau) -\delta -2\frac{C}{Z}\right].
\end{equation}
The stopping power (i.e. $dE/dx$) decreases as the particle energy increases ($\propto \beta^2$). Hence the ionization process is the greatest source of energy loss for particles with small energy.\\
The radiative energy loss process known as \textit{bremsstrahlung} is the dominant source of energy loss by electrons and positrons at energies above $10-100$ MeV, depending on the absorber material. Relativistic electrons and positrons radiate photons as a result of the interaction between Coulomb and the atomic electric fields. The energy spectrum of these photons falls off as $1/E$ ranging till the primary particle energy, but in general most of the photons carry a small part of it.
The process produces (usually small) changes in electron (or positron) direction, contributing to the Coulomb or multiple scattering.\\
At a fixed energy the relative importance of ionization and radiation losses depends on the medium and in particular on its electron density.
This density is in first approximation proportional to the number of protons in the nuclei ($Z$).
The critical energy, i.e. the energy value at which the two processes have equal impact, is roughly inversely proportional to the $Z$ value of the material:
\begin{equation}
    \varepsilon_c = \frac{160\text{ MeV}}{Z + 1.24}.
\end{equation}
An example of energy loss in copper by electron is sketched in Figure \ref{fig:Cu_rad_ion}, where the ionization and radiation contribution are plotted.\\

\begin{figure}
	\centering
	\includegraphics[width=0.8\textwidth]{IMG/Cap2/Cu_rad_ion}
	\caption{Energy losses through ionization and bremsstrahlung by  electrons in copper \cite{PDG_98}.}
	\label{fig:Cu_rad_ion}
\end{figure}

On the other hand, the interaction between photons and matter is mainly affected by three different processes: the photoelectric effect, the Compton scattering and the electron–positron pair production.\\
The photoelectric effect is the process that most likely occurs at low energies. It is characterized by an atom absorbing the photon and emitting an electron. The photoelectric cross section strongly depends on the available number of  electrons,  and  thus  on  the $Z$ value  of  the  absorber  material. In particular it scales with $Z^n$, with the power $n$ between $4$ and $5$. The photoelectric cross section rapidly decrease with greater energies, varying as $E^{-3}$, and the process rapidly loses its impact as the energy increases.

The Compton process is a scattering process where an impinging  photon interact with an atomic electron transferring enough momentum and energy to the struck electron to escape from the atomic Coulomb field. Kinematic variables such as energy transfer and scattering angles can be easily obtained applying the laws of energy and momentum conservation.

Photons in the MeV energy range are absorbed by photoelectric effect only after a sequence of Compton scattering processes, in which the photon energy is reduced step by step in each collision until it is low enough to favour the photoelectric occurrence. In each step, the photon energy loss is:
\begin{equation}
    T = E_{\gamma}\frac{\xi(1 - \cos{\theta})}{1 + \xi(1 - \cos{\theta})}
\end{equation}
where $\xi = E_{\gamma}/m_ec^2$.
The Compton scattering cross section is much less dependent on the $Z$ value than the photoelectric one. It is almost proportional to the number of target electrons in the nuclei. Also in this process the cross section decreases with increasing photon energy, but only with the first power of $E$. Therefore Compton scattering has more impact than photoelectric absorption above a certain threshold energy. 

The pair production process, differently from the previous ones, has a lower threshold under which the effect can not occur. This threshold is twice of the electron rest mass ($2\times 511$ MeV). Above the threshold, a photon can produce an electron-positron pair.
The cross section for pair production rises with the energy reaching an asymptotic value at energies higher then $1$ GeV. For this reason, at high energies, pair production is the most likely process to occur. Meanwhile the dependence from the medium goes, in first approximation, as $Z^2$.\\

Comparing the cross section of the three processes and their dependence to the photon energy it is clear that the photoelectric effect dominates at lower energies, While at the intermediate values the Compton scattering gives the greatest contribution.
At higher energies almost every photon is converted in charged particles through pair production. An example of these contributions is shown in Figure \ref{fig:ph_cross_E}.\\
Knowing the dependence of the cross sections with respect to the $Z$ of the material, ranges of energies where each process dominates can be found and parametrized with the $Z$ value. A representation is sketched in Figure \ref{fig:ph_cross_Z}.

\begin{figure}
	\centering
	\subfloat[][Total cross section of photon in Carbon. Different processes contribution are also separated. \label{fig:ph_cross_E}]{\includegraphics[height=.18\textheight]{IMG/Cap2/ph_cross_E.png}} \quad
	\subfloat[][Energy ranges where different processes dominate with respect to the medium $Z$ value.\label{fig:ph_cross_Z}]{\includegraphics[height=.18\textheight]{IMG/Cap2/ph_cross_Z.png}}
	\caption{Images from \cite{Leo}.}
	%\label{fig:sigma_su_e}
\end{figure}

\subsubsection*{Electromagnetic shower principle}
%Minimal showers may also develop at very low energy of primary particle. Starting for example from a photon of tens of $MeV$, it can eventually produce a electron-positron pair in the calorimeter. The  charged  particles lose their energy in the matter through ionization. When the  positron loses all the kinetic energy, it annihilates with an electron producing two $511$ keV $\gamma$s. These photons are absorbed through the photoelectric effect after a sequence of Compton scattering. During the process, the energy of the primary particle is released to the material by charged particle in ionization processes.\\
At energy values of $1$ GeV and higher, electrons, protons and photons initiate electromagnetic showers in the materials in which they penetrate. At these energies charged particles lose their energy mostly by brehemstralung, the majority of these photons are very soft, and interact with Compton scattering until their absorption through photoelectric effect. Meanwhile the photons with energy more than $5–10$ MeV produce $e^+-e^-$ pairs, which eventually radiate more $\gamma$'s. The process continues till the photon energy is enough to continue the particle multiplication. The shower maximum is defined as the point at which the number of shower particles produced in this particle multiplication process reaches its maximum. The depth inside the absorber associated to the shower maximum increases logarithmically with the energy of the primary particle (see Figure \ref{fig:shower_max}). The longitudinal shower development is described by the radiation length ($X_0$), it is defined as the distance at which the electron (or positron) loses on average $63\%$ $(1-e^{-1})$ of its energy by radiation. Expressing the shower containment in term of $X_0$ is useful to mitigate the dependence of the shower containment on the absorber material.

\begin{figure}
	\centering
	\includegraphics[width=0.7\textwidth]{IMG/Cap2/shower_max.png}
	\caption{The energy deposit as a function of depth, for $1$, $10$, $100$ and $1000$ GeV electron showers developing in a block of copper \cite{Leo}.}
	\label{fig:shower_max}
\end{figure}

Another quantity useful to describe the spatial shower development, in particular the transverse one, is the Molière radius. It is defined in terms of the radiation length and the critical energy:
\begin{equation}
    \rho_M = E_s \frac{X_0}{\varepsilon_c}
\end{equation}
where $E_s$ is defined as $m_c^2\sqrt{4\pi/\alpha} \simeq 21.2$ MeV. This quantity is almost material-independent and, on average, a cylindrical volume with this radius around the shower axis contains $90\%$ of the shower energy. The lateral spread is mainly caused by two effects: at high energy, electrons and positrons are moved away from the shower axis because of the deviation occurring in Compton scattering; photons and electrons are also produced in isotropic processes moving them away from the axis (spread more important in lower energy particles). Also brehemstralung process produces photons with a certain angle, contributing to the shower lateral dimension. Figure \ref{fig:shower_moliere} shows the electromagnetic shower radial profiles at different showering stages.\\

\begin{figure}
	\centering
	\includegraphics[width=0.6\textwidth]{IMG/Cap2/shower_moliere.png}
	\caption{The radial distributions of the energy deposited by $10$ GeV electron showers in copper, at various depths \cite{Leo}.}
	\label{fig:shower_moliere}
\end{figure}

The lateral and longitudinal shower development generated by charged particles and by neutral ones are basically identical except for the initial stages. Electrons start radiating as soon as they enter the calorimeter, instead photons must convert before releasing any energy. Once they start producing electrons and positrons, they can release even more energy than electron induced showers. This behaviour is shown in Figure  \ref{fig:em_start}, where the distribution of the energy fraction deposited in the first $5\ X_0$ by $10$ GeV electrons and photons in lead is plotted.

\begin{figure}
	\centering
	\includegraphics[width=0.7\textwidth]{IMG/Cap2/em_start.png}
	\caption{Distribution of the energy fraction deposited in the first $5\ X_0$ by  $10$ GeV electrons and photons showering in lead. Image from \cite{Wigmans_e_gamma}.}
	\label{fig:em_start}
\end{figure}

\subsection{Hadronic showers} \label{subsec:had_shower}
Introducing the hadronic showers a new degree of complexity arises, indeed, showers produced by hadrons are also affected by strong interactions. This interaction is responsible for:
\begin{itemize}
    \item The production of hadronic secondary particles, most of which ($\simeq 90\%$) are pions. Neutral pions mainly decay in two photons, which develop electromagnetic showers.
    \item The occurrence of nuclear reactions where atomic nuclei release neutrons and protons. The fraction of the shower energy needed to unbind the nucleons does not contribute to any signal formation, and therefore is known as invisible energy.
\end{itemize}

Among the particles in an hadronic shower, neutral pions play an important role. Via their $\pi^0\rightarrow \gamma\gamma$ decay, they start electromagnetic showers over-imposed to the hadronic part. The fraction of energy carried by the induced em showers is called electromagnetic fraction ($f_{em}$).
On average, $f_{em}$ increases with the primary particle energy, since $\pi$'s may also be produced by secondary and higher-order particles: the higher the energy, the more the generations of shower particles, the larger the average em fraction. The  average  electromagnetic fraction has been evaluated to increase with the energy following the power law:
\begin{equation}
    f_{em} = 1 - \left(\frac{E}{E_0}\right)^{k-1}
\end{equation}
where $K\simeq 0.82$ and $E_0$ is a material dependent value of the GeV order. The behaviour in lead and copper is shown in Figure \ref{fig:f_em}.\\
Another contribution to the complexity is given by the variability of this em fraction event by event and its energy dependence.\\

\begin{figure}
	\centering
	\includegraphics[width=0.7\textwidth]{IMG/Cap2/f_em.png}
	\caption{Comparison between the experimental results on the em fraction of pion-induced showers in copper-based and lead-based calorimeters. Image from \cite{fem_par}.}
	\label{fig:f_em}
\end{figure}

Another important difference between em and hadronic showers is the greater spatial profiles of released energy. Analyzing for example several pions showers, peaks of signals are produced at a depth near the one where a $\pi^0$ is generated. However, the neutral pion production can occur in the second or third generation of the shower development releasing energy at different depth. An example of $4$ different showers from $270$ GeV pions is illustrated in Figure \ref{fig:had_start}.\\
The depth of the calorimeter required to contain hadronic showers increases logarithmically with energy, as already seen for em showers, but the large longitudinal fluctuations in the showering starting point make the leakage effect non negligible also in configurations that would contain, on average, $99\%$ of the shower.\\
Laterally, an hadronic shower is more contained if the primary particle energy increases. This is due to the fact that the electromagnetic shower fraction increases with energy and the em showers produced tend to develop laterally closer to the shower axis.
 
 \begin{figure}
	\centering
	\includegraphics[width=0.7\textwidth]{IMG/Cap2/had_start.png}
	\caption{Longitudinal  profiles  for  $4$  different  showers  induced  by  $270$ GeV pions in a lead/iron/plastic-scintillator calorimeter \cite{Wigmans_art_of_cal}.}
	\label{fig:had_start}
\end{figure}

\section{The calorimeter response}
The calorimeter \textit{response} is defined as the average calorimeter signal per unit of deposited energy. The response is usually expressed in terms of number of photoelectrons per GeV or charge (pC) per MeV.
A calorimeter with a constant response is said to be \textit{linear}.
In the following we compare the response of electromagnetic and hadronic calorimeters.\\

\subsection*{Electromagnetic calorimeters}
Electromagnetic calorimeters are detector optimized to absorb em showers. In these showers, all the energy carried by the primary particle is released by a very large number of particles through processes that may generate signals (excitation or ionization in the absorbing medium). The number of particles is on average proportional to the primary energy. The direct consequence is that em calorimeters are in general linear detectors.\\
A non-linear response is usually an indication of instrumental problems. The most common ones are:
\begin{itemize}
    \item leakage effects;
    \item saturation effects of the active components, originated by high localized energy depositions;
    \item recombination of electrons and ions, making the energy carried by the electron undetected.
\end{itemize}

At the same time, the small lateral and longitudinal em shower development allows to design relatively small calorimeters, often used as electromagnetic components in more complex systems.\\

The small dimension also allows to build em calorimeters operating at colliders both of the homogeneous and sampling type.

The main advantage of homogeneous detectors is their excellent energy resolution, achievable because the whole energy of the primary particle is released in the active medium. At the same time, if a well position measurement and particle identification are required, homogeneous calorimeters are not the best choice because of the larger shower dimensions task. 
Homogeneous calorimeter can be classified in four groups:
\begin{itemize}
    \item semiconductor calorimeters;
    \item Cherenkov calorimeters;
    \item scintillator calorimeters;
    \item noble-liquid calorimeters.
\end{itemize}

On the other hand, sampling calorimeters have in general a worse energy resolution due to the presence of an absorber material (copper, iron, lead or uranium) that does not contribute to the signal. For electromagnetic sampling calorimeters, typical resolution values are in the range of $5-20 \% / \sqrt{E(\text{GeV})}$. However, they are relatively simple to segment longitudinally and laterally due to their sampling structure often helping to achieve a better spatial resolution.

\subsection*{Hadronic calorimeters}
Hadronic calorimeters are detector optimized to absorb hadronic showers. At collider experiments, to limit the detector dimensions they are always of the sampling type. The calorimeter response is a mixture between the response to the em ($e$) and non-em ($h$) components. The ratio $e/h$ classifies calorimeters in three categories: \textit{compensating}, if $e/h=1$; \textit{undercompensating}, if $e/h>1$; \textit{overcompensating}, if $e/h<1$. Hence, the total calorimeter response (to hadrons) is a combination of the two:
\begin{equation}
    \pi = f_{em} \cdot e + (1- f_{em}) \cdot h.
\end{equation}
Since the average $f_{em}$ value, as already seen, increases with the energy, the response of a non-compensating calorimeter ($e/h \neq  1$) is not constant and non-compensating calorimeters are intrinsically non-linear detectors.\\
The $e/h$ value cannot be directly measured. However, it is usually derived from the $e/\pi$ ratios, measured at various energies. The relationship between $e/\pi$ and $e/h$ is:
\begin{equation}
    \frac{e}{\pi} = \frac{e/h}{1 - f_{em}(1-e/h)},
\end{equation}
where $f_{em}$ is the energy dependent average em fraction. A calorimeter for hadron detection does not present a linear response with respect to the primary particle due to the energy dependence of the em fraction. The relation is represented in Figure \ref{fig:compensation} for compensating and non-compensating calorimeters.\\

 \begin{figure}
	\centering
	\includegraphics[width=0.65\textwidth]{IMG/Cap2/compensation.png}
	\caption{Relation between the calorimeter response ratio to em and non-em energy deposits, $e/h$, and the measured $e/\pi$ signal ratios \cite{Wigmans_art_of_cal}.}
	\label{fig:compensation}
\end{figure}

The basic reason for different response for em and hadronic showers lies in the fact that in the absorption of hadronic showers, a significant fraction of the energy is \textit{invisible} not contributing to the calorimeter signal. The main source of this energy is the energy used to release nucleons from nuclei, the invisible energy.\\
To design a linear calorimeter various compensation methods have been developed. There are two main approaches to obtain $e/h = 1$: one is reducing the electromagnetic response and the other one is to increase the non-em response.\\
The most effective way to reduce the em response of a sampling calorimeter is to use an high-$Z$ material as absorber. The concept is based on the fact that low energy photons release their energy via the photoelectric effect that is highly $Z$ dependent ($Z^5$). For this reason, using high-$Z$ absorber materials causes an energy deposition in the absorber making their energy less likely to be detected in the active elements.\\
The other strategy is known as compensation by neutrons signal boosting or by Signal Amplification through Neutron Detection (SAND). It consists in increasing the non-em response taking advantage of the kinetic energy transported by neutrons. The good correlation between the invisible energy and the kinetic energy of neutrons gives the possibility of an event-by-event correction boosting the signals generated by the neutrons. The most likely process that occur at low energies for a neutron is elastic scattering on a nucleus target. The fraction of energy transferred in this process is on average $f_{\text{elastic}} = 2A / (A +1)^2$ (where $A$ is the mass number of the target). Considering that hydrogen maximize this fraction it is the element that must be present in the active medium to produce the non-em response increment. With a proper fine tuning of the sampling fraction, a $e/h$ ratio of $1$ can be reached.
In short, compensation through SAND can be achieved in sampling calorimeters with active materials containing hydrogen and a precisely tuned sampling fraction.\\
In the IDEA dual-readout calorimeter, a different compensation strategy is adopted the dual-readout compensation that will be described in Section \ref{sec:DRComp}.

\subsection{Fluctuations}
Event by event fluctuations in the calorimeter response determine the detector energy resolution. The energy resolution is quantified as the reconstructed energy distribution standard deviation divided by the mean value ($\sigma/E$).\\
In em calorimeters, four are the main sources of response fluctuation:
\begin{itemize}
    \item sampling fluctuations;
    \item signal quantum fluctuations;
    \item shower leakage fluctuations;
    \item instrumental fluctuations.
\end{itemize}
Sampling effects are affected both by the sampling fraction (i.e. the ratio of the active material on the total) and the sampling frequency (the thickness of the layers). The sampling fraction is  defined as:
\begin{equation}
    f_{samp} = \frac{E_{active}}{E_{passive}+E_{active}}
\end{equation}
where $E_{active}$ and $E_{passive}$ are the energies deposited in the active and passive part by an incident minimum ionizing particle (mip).
This type of fluctuation is dominated by the Poisson statistics, hence it contributes with a term proportional to $1/\sqrt{E}$ to the energy resolution.
In em calorimeter with non-gaseous active material the sampling contribution follows the empirical law:
\begin{equation}
    \frac{\sigma}{E} = \frac{2.7\% \sqrt{d/f_{samp}}}{\sqrt{E}}
\end{equation}
where $d$ is the thickness of the active material layer measured in mm.\\
Fluctuation effects due, for example, to photon statistics in light emitting active materials are grouped under the label of signal quantum fluctuations. Also these fluctuations follow the rules of Poisson statistics, with the constrain of uncorrelated separate contributions. Another contribution that scales with $E^{-1/2}$ is added to the relative energy resolution expression.\\
In leakage fluctuations there are three possibilities: longitudinal, lateral and albedo. These fluctuation are highly dependent to the geometry and the material of the calorimeter. For this reason their contribution to the resolution does not have a precise energy dependent form. The typical solution to numerically evaluate their effect is via Monte Carlo simulation of the detector.\\
Finally, instrumental fluctuations are the ones provided, for example, by electronic noise or geometrical inhomogeneities. The electronic noise contribution is usually described with a term that scales with $1/E$ because it is largely independent of the shower energy. Meanwhile fluctuations induced by structural inhomogeneities depend on the shower position and might be energy dependent.\\
Typically, these contribution are uncorrelated and have to be added in quadrature to obtain the total resolution. As seen, different contribution with different energy dependence have to be considered. The consequence is that different effects dominate the energy resolution in different energy ranges. These behaviour have to be considered in the design of the calorimeter in order to optimize the resources.\\

All the sources described affect also hadronic calorimeters. For example, sampling fluctuation has a higher impact in hadronic showers due mainly to the lower average number of particle that release energy in the hadronic ones. 
At the same time, in hadronic detectors, resolution is affected also by other fluctuation effects.\\
%The introduction of quantities such as $f_{em}$ in hadronic calorimeters produces new sources of fluctuations.
In non-compensating calorimeters, event-by-event fluctuation in em component affect the detector response. It contributes to the relative energy resolution through a term scaling as $c E^{-0.28}$.
\begin{equation}
    \frac{\sigma}{E} = \frac{a_1}{\sqrt{E}} \oplus c E^{-0.28}
\end{equation}
Up to $400$ GeV, this law runs almost parallel to an energy resolution in which only the stochastic term is included ($\sigma/E = a/\sqrt{E}$). For this reason, the resolution of non-compensating calorimeter is often described as:
\begin{equation}
    \frac{\sigma}{E} = \frac{a_2}{\sqrt{E}} + b.
\end{equation}
A useful solution that remove these fluctuations is to design a compensating calorimeter, most often improving the energy resolution.\\
%Finally, the calorimeter resolution is also affected by invisible energy fluctuations. To evaluate the impact of these fluctuations the correlation between $f_{em}$ and $f_{inv}$ has to be known, and in particular if they are correlated or not. If they are uncorrelated the contribution to the resolution has to be added in quadrature. Moreover, in compensating calorimeter, the contribution of invisible energy fluctuations are different depending on the techniques used to achieve compensation.

\section{Dual-readout compensation}\label{sec:DRComp}
All the benefits obtained by compensating calorimeters are achieved by sampling em and non-em components in hadronic showers with the same response ($e=h$). Dual-readout calorimetry is a method that measures the electromagnetic fraction on an event-by-event basis and applies a correction, reaching $e/h = 1$ value \cite{DR_Wig}. 

\subsection{Working principles}
This technique is based on the concept that the em component is mostly brought by relativistic electrons and positrons, while the majority of non-em one is carried by non-relativistic particles. Hence, collecting the Cherenkov signal produced by an hadronic shower is almost equivalent to sampling the em component, thus measuring the electromagnetic fraction. A second simultaneous signal, typically from scintillators, is recorded to provide information on the total event ionizing energy.\\ %Thanks to this second signal, the non-em component can be evaluated starting from the $f_{em}$ already obtained.\\

Let's consider scintillation ($S$) and Cherenkov ($C$) signals produced by an hadronic shower developing in a dual-readout calorimeter. The mean values of these signals are calibrated with an electron beam of known energy $E$ so that, for em showers, $\expval{S} = \expval{C} = E$. The hadronic signals can be written (for each event) as:
\begin{align}
    S &= E \left[f_{em} + (h/e)_S (1-f_{em}) \right], \label{eq:S} \\
    C &= E \left[f_{em} + (h/e)_C (1-f_{em}) \right], \label{eq:C}
\end{align}
where $h/e$ quantifies the two different degree of non-compensation for the two signals. Considering that $h/e$ are measurable quantities and are assumed to be constants, the ratio $C/S$ is also energy independent:
\begin{equation}
    \frac{S}{C} = \frac{f_{em} + (h/e)_S (1-f_{em})}{f_{em} + (h/e)_C (1-f_{em})}. \label{eq:em_frac}
\end{equation}
This expression can straightforwardly give an evaluation of the em fraction:
\begin{equation}
    f_{em} = \frac{(h/e)_C-(C/S)(h/e)_S}{(C/S)\left[1-(h/e)_S\right]-\left[1-(h/e)_C\right]},
\end{equation}
once the $h/e$ values are known.\\
%allowing the rewriting of \ref{eq:S} and \ref{eq:C} as
%\begin{align}
%    S/E &= (h/e)_S + f_{em}\left[1-(h/e)_S\right], \label{eq:S2} \\
%    C/E &= (h/e)_C + f_{em}\left[1-(h/e)_C\right]. \label{eq:C2}
%\end{align}

An example of a $C/E$ vs $S/E$ scatter plot is shown using signals from RD52 DR calorimeter for electrons, pions and protons showers. As expected, data generated from electrons are always around the point $[1,1]$ (as $f_{em} = 1$). On the other hand an hypothetical event with only the non-em component would be at $[(h/e)_S,(h/e)_C]$. Signals from pions and protons showers are clustered along a straight line linking these two points.
%From this geometric construction it is clear that the slope of this line (the $\theta$ angle) is energy independent because it is determined only by the two $e/h$ values.
Thanks to the fact that $\theta$ is energy and particle independent, it is possible to define the parameter:
\begin{equation}
    \chi = \frac{1-(h/e)_S}{1-(h/e)_C} = \cot{\theta},
\end{equation}
this parameter can be estimated with test beam data.
Finally, event-by-event, the energy of each hadron shower, corrected for the $f_{em}$ value, can be reconstructed using the $S$ and $C$ signals as:
\begin{equation}
    E = \frac{S - \chi C}{1 - \chi}.
\end{equation}

\begin{figure}
	\centering
	\includegraphics[width=0.65\textwidth]{IMG/Cap2/theta_DR.png}
	\caption{Relation between the calorimeter response ratio to em and non-em energy deposits, $e/h$, and the measured $e/\pi$ signal ratios \cite{DR_Theta}.}
	\label{fig:theta_DR}
\end{figure}

The ideal DR calorimeter would able to identify the Cherenkov signal as the exact electromagnetic component ($h/e = 0$) being a direct measurement of $f_{em}$. The worst DR calorimeter instead would have $(h/e)_C = (h/e)_S$, in this case the two signals would sample the two components (em and non-em) with the same response giving no information about the $f_{em}$. Therefore, the best DR calorimeter is the one with the lower $\chi$ parameter corresponding to the lower $(h/e)_C$ and the higher $(h/e)_S$ values possible. 

\subsection{Dual-readout sampling calorimeters}
At present a few dual-readout calorimeter have been built and tested, and more project are under study.\\
The first practical demonstration of the dual-readout method was achieved by a R\&D study for the Advanced Cosmic Composition Experiment at the Space Station (ACCESS) \cite{ACCESS}. The prototype had a depth of $1.4\ \lambda_{int}$ and was equipped with both plastic-scintillator ($S$) and quartz ($Q$) optical fibres to measure and collect the scintillation and Cherenkov light, respectively. With this detector, the $Q/S$ signals ratio was found to represent a good event-by-event measurement of the shower energy fraction carried by $\pi^0$.\\

Later, a $10\ \lambda_{int}$ deep calorimeter known as Dual-REAdout Module (DREAM) was built and tested \cite{DREAM1, DREAM2}. 
The detector is composed by $5580$ basic element represented in Figure \ref{fig:DREAM}. These are $200$ cm long, hollow, extruded copper rod of $4\times 4$ mm$^2$ cross section in which $3$ scintillating and $4$ Cherenkov fibres are inserted.
\begin{figure}
	\centering
	\includegraphics[width=0.65\textwidth]{IMG/Cap2/DREAM.png}
	\caption{The DREAM calorimeter layout \cite{DREAM}.}
	\label{fig:DREAM}
\end{figure}

Figure \ref{fig:SC_Dream_sig} shows the energy distributions for $100$ GeV $\pi^-$ by the two fibre types, after a calibration at the electromagnetic scale. The distributions are peaked at considerably lower values than the one corresponding to electrons.
Populating a scatter plot with Cherenkov and scintillation signals for each event, the correlation between them is evident as shown in Figure \ref{fig:theta_DREAM}.
The two $h/e$ ratios for Cherenkov and scintillator DREAM structures were measured to be $0.21$ and $0.77$, respectively. The expression of the em fraction \ref{eq:em_frac} becomes:
\begin{equation}
    f_{em} = \frac{0.21-0.77(C/S)}{(C/S)\left[1-0.77\right]-\left[1-0.21\right]}.
\end{equation}

Being able to evaluate $f_{em}$ event-by-event open the possibility to correct the signals for the effects of non-compensation.
In Figure \ref{fig:fem_subsamp} the overall Cherenkov signal distribution for $100$ GeV $\pi^-$ (a) is shown, where the asymmetry from $f_{em}$ fluctuations is evident. In Figure \ref{fig:fem_subsamp}(b) the events are grouped in subsamples selected on the basis of their $f_{em}$ value.
Each one of these subsamples reproduce a certain region of the overall signal distribution, with the average value that increases with $f_{em}$ as expected. This representation gives an idea of the effectiveness of the method.\\
In this process, the energy resolution improved, the signal distribution became much more Gaussian and, most importantly, the hadronic energy was, on average, correctly reproduced, i.e. liearity greatly improved.\\

\begin{figure}
	\centering
	\includegraphics[width=0.65\textwidth]{IMG/Cap2/SC_Dream_sig.png}
	\caption{Signal distributions for $100$ GeV $\pi^-$ recorded by the scintillating (a) and the Cherenkov (b) fibers. \cite{Wigmans_art_of_cal}.}
	\label{fig:SC_Dream_sig}
\end{figure}

\begin{figure}
	\centering
	\includegraphics[width=0.65\textwidth]{IMG/Cap2/theta_DREAM.png}
	\caption{The $C-S$ scatter plot showing the correlation between the two signals. The signals are expressed in the same units used to calibrate the calorimeter (em GeV) \cite{DREAM2}.}
	\label{fig:theta_DREAM}
\end{figure}

\begin{figure}
	\centering
	\includegraphics[width=0.65\textwidth]{IMG/Cap2/fem_subsamp.png}
	\caption{Cherenkov signal distribution for $100$ GeV $\pi^-$ (a) and distributions for subsamples of events selected on the basis of the measured $f_{em}$ value, using the $Q/S$ method (b). Image from \cite{Wigmans_art_of_cal}.}
	\label{fig:fem_subsamp}
\end{figure}

The DREAM results, together with other studies, confirmed the feasibility of the dual-readout compensation method, leading to the IDEA dual-readout calorimetry project.
		
		%--------CAPITOLO 3--------
		\chapter{Calorimetry and dual-readout}
Calorimetry is an important detection principle in particle physics.
Originally developed with astrophysical purpose for cosmic-ray studies, this method refers to the detection of particles and the measurement of their properties, using blocks of instrumented material.
It was developed and perfected for accelerator-based particle physics experimentation primarily in order to measure the energy of particles. 
In these blocks, particles are fully absorbed and their energy transformed into a measurable quantity.\\
The incident particle interact with the detector (through electromagnetic or strong processes) producing a shower of secondary particles with progressively degraded energy.
The energy deposited by the charged particles of the shower in the active material of the calorimeter, which can be detected in the form of charge or light, is used to measure the energy of the incident particle.
Typical processes suitable to detect this energy are: ionization of the medium, scintillation light and the Cherenkov light produced by relativistic particles.\\

Calorimeters can be divided into two categories depending on the type of shower they are optimized to detect: electromagnetic calorimeters, used mainly to measure electrons and photons through their electromagnetic interactions (e.g., bremsstrahlung, pair production), and hadronic calorimeters, used to measure mainly hadrons through their strong and electromagnetic interactions.\\
Another classification can be made according to their construction technique defining sampling calorimeters and homogeneous calorimeters.\\
Homogeneous calorimeters are built of one type of material that performs both the main tasks: degrade the energy of the incident particles and provide the detectable signal.\\
Sampling calorimeters, instead, consist of alternating layers of an absorber, a dense material used to perform energy degradation, and an active medium that generate the signal.\\

Calorimeters are attractive in high-energy particle physic field for various reasons:
\begin{itemize}
		\item In most cases the calorimeter energy resolution improves with energy as $1/\sqrt{E}$, where $E$ is the energy of the incident particle. Therefore calorimeters are very well suited to high-energy physics experiments.
		\item Calorimeters are sensitive to all types of particles, charged and neutral (e.g., neutrons). Also neutrinos and their energy can be indirectly detected can even provide indirect detection of neutrinos and their energy through the measurement of the event missing energy.
		\item They are versatile detectors. They can be used to determine the shower position and direction, to perform particle identification, to measure the arrival time of the particle, or even to provide fast signals useful in trigger purpose.
		\item They are space and therefore cost effective. Because the shower length increases only logarithmically with energy, the detector thickness needs to increase only logarithmically with the energy of the particles.
\end{itemize}

This chapter describes the physics behind both the electromagnetic and hadronic shower developments, provides a basic description of the energy response of these detectors and introduces the particular technique of the dual-readout, a modern concept of calorimeter that has the quality of overcome the non-compensating problem producing both electromagnetic and hadronic showers measuring two different type of signal simultaneously (Cherenkov and scintillation light).\\

\section{Physics of shower development}
When a particle traverses matter, it will generally interact and lose (a fraction of) its en-ergy in doing so. The medium is excited in this process, or heated up, whence the term calorimetry.
The interaction processes that play a role depend on the energy and the nature of the particle.
They are the result of the electromagnetic (em), the strong and, more rarely, the weak forces reigning between the particle and the medium’s constituents.
In this chapter, the various processes by which particles lose their energy when traversing dense matter and by which they eventually get absorbed, are described.
We also discuss shower development characteristics, the effects of the electromagnetic and strong interactions, and the consequences of differences between these interactions for the calorimetric energy measurement of electrons and hadrons, respectively.

\subsection{Electromagnetic Showers}
aaa

\subsection{Hadronic showers}
aaa

\section{Energy response of calorimeters}
aaa

\subsection{Homogeneous calorimeters}
aaa

\subsection{Sampling calorimeters}
aaa

\subsection{Compensation}
aaa

\section{Dual-readout calorimetry}
aaa

\subsection{Working principles}
aaa

\subsection{Experiments}
		
		%--------CAPITOLO 4--------
		\chapter{IDEA DR calorimeter project} \label{chap:Idea_project}
aaa

		%--------CAPITOLO 5--------
		\chapter{IDEA DR calorimeter full simulation}
As already said, the project described in chapter \ref{chap:Idea_project} is an on-going production and has to be supported by simulation.
With this goal, a dual-readout calorimeter full simulation has been developed allowing to generate data and monitor the whole process from the collision on the interaction point to the digitized signal produced by SiPMs.\\

The chapter presents a description of the simulation structure. The section \ref{sec:Sim_struc} describes in details the simulation dividing it in two main Monte Carlo processes:
\begin{itemize}
	\item the calorimeter simulation, coded in GEANT4;
	\item the SiPM response digitization ("pySIPM"), coded in Python.
\end{itemize}

Later, the performances obtained will be shown. The temporal behavior, the SiPM saturation effect and the energy resolution will be described in section \ref{sec:Sim_perf}.\\

The second half of the chapter treats of the possibility of simple particle identification using neural network structures.\\
In section \ref{sec:NN_waveform} neural networks working on digitized waveforms are described. The aim of these neural network is to correctly distinguish waveforms generated by electrons ($e^-$) or pions ($\pi^-$) in a range of energy from $20$ to $80$ GeV.\\
The last section (sec.\ref{sec:NN_img})  exposes another type of neural networks. These have the purpose of identify if signal are generated from photons ($\gamma$) or neutral pions ($\pi^0$) analyzing the spazial pattern of energy released in the calorimeter.

\section{Simulation structure} \label{sec:Sim_struc}
aaa

\subsection{Calorimeter simulation} \label{subsec:Sim_cal}
aaa

\subsection{SiPM response digitization} \label{subsec:Sim_SiPM}
aaa

\section{Simulation performances} \label{sec:Sim_perf}
aaa

\subsection{Different configurations} \label{subsec:SiPM_conf}
aaa

\subsection{Time studies} \label{subsec:Time}
aaa

\subsection{Saturation effect} \label{subsec:Sat_effect}
aaa

\subsubsection{Occupancy effect and Energy loss}
Studies of the occupancy effect are important preliminary studies that give knowledge about the information loss in the detection process.\\

\subsection{Digitization impact on energy resolution} \label{subsec:E_res}
aaa

\section{Neural Network: Particle ID on waveform} \label{sec:NN_waveform}
aaa

\subsection{Configuration}
aaa

\subsection{Performances}
aaa

\section{Neural Network: Particle ID on imaging} \label{sec:NN_img}
aaa

\subsection{Configuration}
aaa

\subsection{Performances}
aaa
		
		%-------CONCLUSIONE--------
		\chapter*{Conclusion}
\addcontentsline{toc}{chapter}{Conclusion} 
aaa
		
	\end{mainmatter}
	
	%_________________________CORPO FINALE
	\begin{backmatter}
		\input{Ringraziamenti}
		
		%--------BIBLIOGRAFIA------
		\begin{thebibliography}{99}
			\addcontentsline{toc}{chapter}{Bibliography}
			
			%\bibitem{stringa}autore libro ecc.		lo richiamo con \cite{stringa}
			\bibitem{sk1} Y. Fukuda et al., Phys. Rev. Lett. 81 (1998) 1158-1162.
			\bibitem{Snell} Y. Fukuda et al., Phys. Rev. Lett. 81 (1998) 1158-1162.
			\bibitem{digitizer} Y. Fukuda et al., Phys. Rev. Lett. 81 (1998) 1158-1162.
			\bibitem{SiPM_lineup} Hamamatsu SiPMs lineup \url{https://www.hamamatsu.com/eu/en/product/optical-sensors/mppc/mppc_mppc-array/all_products/index.html}		
		\end{thebibliography}
		
	\end{backmatter}
	
	
\end{document}