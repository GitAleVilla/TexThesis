\chapter{Silicon Photomultipliers}
SiPMs, also referred to as Multi Pixel Photon Counters (MPPCs) or Geiger mode Avalanche Photo-Diodes (G-APDs), are solid state light sensors featuring high internal gain, single-photon sensitivity, unprecedented photon number resolving capability, high Photon Detection Efficiency (PDE) and dynamic range, excellent time resolution, low bias voltage and magnetic field insensitivity. They represent an interesting alternative to the vacuum-based technologies thanks to their compactness, ruggedness, low cost and high-volume production capability. SiPMs benefit from the fast evolution of the silicon technology and the investment of different companies in terms of high quality mass production and design flexibility, enabling also a class of new applications spreading from the industrial and consumer technology sectors to fundamental research,  as High Energy Physics (HEP).  In this chapter, an overview of the main properties of these detectors will be provided, together with the description of the most important SiPM parameters and with some examples of applications in particle physics detectors.

\section{Working principles}
SiPM is a high density (up to $10^4/mm^2$) matrix of Single-Photon Avalanche Diodes (SPADs), called pixels1or cells, arranged on a common substrate with a common load and connected in parallel to a single readout output.   The photo-diodes are usually p/n junctions, designed to be biased few volts above breakdown voltage ($V_{Bk}$) in limited Geiger-Müller regime. Thanks to the high electric field in the depletion region, initial charge carriers generated by an absorbed photon (or by thermal effects) trigger an exponential charge multiplication by impact ionization. The process is stopped when the current spike across the quenching resistance induces a drop in the operating voltage. Thanks to its intrinsic charge amplification mechanism, requiring only a single carrier to detect the light pulse,  SPAD features high single photon sensitivity. As a first order approximation, each diode provides the same signal with $\sim 10^6$ gain despite of the number of primary carriers generated.   Thus the SiPM can be seen as a collection of binary pixels: by counting the number of fired cells it can provide information about the intensity of the incoming light.  The typical SiPM cells size ranges between $10\times 10\ \mu m^2$ and $100\times 100\ \mu m^2$, while the SiPM total areas from $1\times 1\ mm^2$ up to $6\times 6\ mm^2$ are available.

\subsection*{Single photon avalanche diode}
Fig. represents an illustrative picture of the SPAD doping structure: applying a reverse bias, the p-side is fully depleted and generates an electric field similar to that shown on the right side of the figure. The thin ($0.1-1.5\ \mu m$) n+ side receives the impinging photons trough a window, while the multiplication region of high electrical field (in the order of few 105V/cm) is placed between the n+ and players and is thin $0.7-0.8\ \mu m$. Its nearly uniform field allows the electron-hole pairs separation and drifts them towards the n+ and p+ sides, respectively.When the drifting electron reaches the junction volume, it is accelerated by the high field and initiates the avalanche by impact ionization.

\subsection*{Electrical model}
The SPAD can be modelled as the circuit of fig.4.4:  a parallel connection between the inner depletion region capacitance ($C_D$) and the internal space-charge resistance of the avalanche region ($R_S$). Each single photodiode has in series a quenching resistor ($R_Q$). The current flowing through the switch is called $I_{INT}$, while the external current is the $I_{EXT}$.\\
Before photon detection, the switch is open (OFF condition) and $C_D$ is charged to the bias voltage ($V_{Bias}$). Upon an avalanche discharge, the switch is closed (ON condition) and the $C_D$ capacitance discharges through the resistor $R_S$ down to the breakdown voltage with a time constant $\tau_D=C_D R_S$. $I_{INT}$ decreases exponentially from $(V_{Bias}-V_{Bk})/R_S$ while $I_{EXT}$ increases with the same time constant $\tau_D$. Both currents tend to the asymptotic value: $(V_{Bias}-V_{Bk})/R_Q$. In this phase the diode current is low (less than $10-20\ \mu A$), and a statistical fluctuation can quench the avalanche bringing the number of carriers in the multiplication region to zero.
The switch is again open and the circuit returns in its initial OFF configuration. $I_{INT}$ goes suddenly to zero, while $I_{EXT}$ decreases exponentially. The capacitance $C_D$ starts recharging to its original $V_{Bias}$ with a time constant (called cell recovery time) $\tau_r=C_D R_Q$. Finally the SPAD is ready to detect a new photon. The easiest way to obtain a SPADs array is to connect them in parallel (building a SiPM).The SiPM current signal is the sum of the $I_{EXT}$ coming from each SPAD fired by a photon and its amplitude and/or the signal charge are proportional to the impinging light intensity.\\
Fig. represents the current flowing during a discharge of a SPAD and its output signal time development.\\
Thanks to the very thin depletion layer and the extremely short avalanche discharge development duration, the SPAD response is intrinsically very fast. The signal rise time ($< 500\ ps$) follows an exponential function with $\tau_D$ time constant and does not depend on the number of fired cells and bias voltage applied. The falling signal is an exponential with time constant $\tau_r$. The total recovery time ranges between $20$ and $250\ ns$, accordingly to the quenching resistance value and to the cell size.\\


\section{SiPM Response}
aaa

\section{Noise effects}
aaa

\subsection{Dark Count Rate}
aaa

\subsection{After-Pulse}
aaa

\subsection{Optical Cross-Talk}
