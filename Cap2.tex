\chapter{Calorimetry and dual-readout}
Originally invented for cosmic-ray studies, calorimetry refers to the detection of particles and the measurement of their properties, using blocks of instrumented material.
They can be divided in two different types: homogeneous and sampling calorimeters.
Homogeneous calorimeters are composed by only one, high density material that provides both the functions of absorbing the particles and detecting the signal produced in this process.
Instead, in sampling calorimeters the particle absorption is performed by a material (passive medium) while the signal generation is exercised by a different one (active medium).
In a calorimeter,particles to be measured are fully absorbed and their energy is transformed into a measurable quantity (like charge or light) [59].
Typical processes suitable to detect this energy are: ionization of the medium, scintillation light and the Cherenkov light produced by relativistic particles.
This chapter describes the physics behind both the electromagnetic and hadronic shower developments, provides the most important definition adopted in this field and introduces the fundamental energy resolution concept.\\


Calorimetry  is  an  ubiquitous  detection  principle  in particle  physics.
Originally  invented  for  the  study  of cosmic-ray phenomena, this method was developed and perfected  for  accelerator-based  particle  physics  experimentation  primarily  in  order  to  measure  the  energy  of electrons,   photons,   and   hadrons.
Calorimeters   are blocks of instrumented material in which particles to be measured  are  fully  absorbed  and  their  energy  transformed  into  a  measurable  quantity.
The  interaction  of the incident particle with the detector (through electromagnetic or strong processes) produces a shower of secondary  particles  with  progressively  degraded  energy.
The  energy  deposited  by  the  charged  particles  of  the shower  in  the  active  part  of  the  calorimeter,  which  can be  detected  in  the  form  of  charge  or  light,  serves  as  a measurement of the energy of the incident particle.
Calorimeters can be broadly divided into electromagnetic calorimeters, used mainly to measure electrons and photons through their electromagnetic interactions (e.g., bremsstrahlung, pair production), and hadronic calorimeters,  used  to  measure  mainly  hadrons  through  their strong  and  electromagnetic  interactions.
They  can  be further  classified  according  to  their  construction  technique   into   sampling   calorimeters   and   homogeneous calorimeters.
Sampling  calorimeters  consist  of  alternating  layers  of  an  absorber,  a  dense  material  used  to  degrade the energy of the incident particle, and an active medium  that  provides  the  detectable  signal.
Homogeneous calorimeters, on the other hand, are built of only one  type  of  material  that  performs  both  tasks,  energy degradation and signal generation.
Today particle physics reaches ever higher energies of experimentation, and aims to record complete event information.
Calorimeters  are  attractive  in  this  field  for various reasons:
\begin{itemize}
		\item In contrast with magnetic spectrometers, in which the momentum  resolution  deteriorates  linearly  with  the particle momentum, in most cases the calorimeter energy resolution improves with energy as $\/\sqrt{E}$,  where $E$ is  the  energy  of  the  incident  particle.  Therefore calorimeters are very well suited to high-energy physics experiments.
		\item In contrast with magnetic spectrometers, calorimeters are sensitive to all types of particles, charged and neutral  (e.g.,  neutrons).  They  can  even  provide  indirect detection  of  neutrinos  and  their  energy  through  a measurement of the event missing energy.
		\item They are versatile detectors. Although originally conceived  as  devices  for  energy  measurement,  they  can be  used  to  determine  the  shower  position  and  direction, to identify different particles (for instance, to distinguish electrons and photons from pions and muons on  the  basis  of  their  different  interactions  with  the detector), and to measure the arrival time of the particle. Calorimeters are also commonly used for trigger purposes,  since  they  can  provide  fast  signals  that  are easy to process and to interpret.
		\item They  are  space  and  therefore  cost  effective.  Because the shower length increases only logarithmically with energy,  the  detector  thickness  needs  to  increase  only logarithmically  with  the  energy  of  the  particles.  In contrast, for a fixed momentum resolution, the bending power $BL^2$ of a magnetic spectrometer (where $B$ is the magnetic field and $L$ the length) must increase linearly with the particle momentum $p$.
\end{itemize}
Besides perfecting this technique to match the physics potential  at  the  major  particle  accelerator  facilities,  remarkable  extensions  have  been  made  to  explore  new energy  domains.
Low-temperature  calorimeters,  sensitive to phonon excitations, detect particles with unprecedented energy resolution and are sensitive to very-low-energy deposits which cannot be detected in conventional  devices.
The  quest  to  understand  the  origin,  composition,  and  spectra  of  energetic  cosmic  rays has  led  to  imaginative  applications  in  which  the  atmosphere  or  the  sea  are  instrumented  over  thousands  of cubic kilometers.
A regular series of conferences (CALOR, 2002) and a comprehensive recent monograph (Wigmans, 2000) testify to the vitality of this field.\\



\section{Physics of shower development}
aaa

\subsection{Electromagnetic Showers}
aaa

\subsection{Hadronic showers}
aaa

\section{Energy response of calorimeters}
aaa

\subsection{Homogeneous calorimeters}
aaa

\subsection{Sampling calorimeters}
aaa

\subsection{Compensation}
aaa

\section{Dual-readout calorimetry}
aaa

\subsection{Working principles}
aaa

\subsection{Experiments}