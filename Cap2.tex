\chapter{Calorimetry and dual-readout}
Originally invented for cosmic-ray studies, calorimetry refers to the detection of particles and the measurement of their properties, using blocks of instrumented material.
They can be divided in two different types: homogeneous and sampling calorimeters.
Homogeneous calorimeters are composed by only one, high density material that provides both the functions of absorbing the particles and detecting the signal produced in this process.
Instead, in sampling calorimeters the particle absorption is performed by a material (passive medium) while the signal generation is exercised by a different one (active medium).
In a calorimeter,particles to be measured are fully absorbed and their energy is transformed into a measurable quantity (like charge or light) [59].
Typical processes suitable to detect this energy are: ionization of the medium, scintillation light and the Cherenkov light produced by relativistic particles.
This chapter describes the physics behind both the electromagnetic and hadronic shower developments, provides the most important definition adopted in this field and introduces the fundamental energy resolution concept.\\

\section{Electomagnetic showers}
aaa

\subsection{Shower development}
Electromagnetic showers are produced by electrons, positrons and gammas.
Contrary to their apparently complex phenomenology, the QED processes that play a role in an em shower development are few and well understood.
More-over the main shower features can be parametrised in few equations.
Fig.2.1 shows the average energy lost by electrons in lead (left plot) and the interaction cross section of photons (right plot) as a function of energy.
For low energies, electrons (and positrons) lose their energy primarily by ionisation and thermal excitation, while photons through Compton and photoelectric effect.
At a higher energy regime ($E > 10\ MeV$) the main mechanism for energy loss by electrons is bremsstrahlung and photon interaction is governed by pair production.
Finally,at energies of $1\ GeV$ or higher, both these processes become roughly energy independent and electrons and photons initiate em showers on the calorimeter: $e$ produces secondary photons (by bremsstrahlung) and $\gamma$ secondary $e^-\ e^+$ pairs.
This mechanism is replicated by these secondary particles giving rise to a shower of particles that progressively lose their energy.
The number of particles constituting the shower increases until the electron component energy decreases below the so called critical energy $\varepsilon$.
It is defined as the energy at which both ionization and bremsstrahlung losses become equal.
It is roughly inversely proportional to the $Z$ value of the absorber material and (for solids) is approximately given by:
\begin{equation}
	\varepsilon = \frac{610\ MeV}{Z + 1.24}.
\end{equation}

\subsection{Energy resolution}
aaa

\section{Hadronic showers}
aaa

\subsection{Shower development}
aaa

\subsection{Energy resolution}
aaa

\section{Dual-readout calorimetry}
aaa

\subsection{Working principles}
aaa

\subsection{Experiments}