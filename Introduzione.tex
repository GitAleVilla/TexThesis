\chapter{Introduction}
The discovery of the Higgs boson in 2012 at CERN has opened a number of new studies to be inspected in particle physics. Its relatively small mass allows to produce it in high luminosity electron-positron colliders.
The Higgs boson is a crucial part of the biggest mysteries of modern particle physics and the precise measure of its (with $W$ and $Z$ bosons) properties will provide extremely usefull information in describing the Standard Model and in exploring physics bejond it (BSM). 
\\

In the first chapter a short introduction describes the main goals in modern particle phisycs. The future accelerators design are then shown pointing out their capabilities to improve our knowledge.
Finally the IDEA detector concept is outlined, being one of the baseline multi-detector concepts.
\\

The second chapter gives a theoretical background on calorimetry, starting from the interaction between particles and matter and the production of particle showers. Then a description of calorimeter is provided, distinguishing electromagnetic and hadronic calorimeter. Eventually, the dual readout compensation technique, cornerstone of the IDEA calorimeter, is introduced along with its strengths and weaknesses.
\\

The third chapter is dedicated to Silicon PhotoMultipliers. They are photo-detectors that are studied to be coupled to the IDEA Dual-Readout calorimeter. The working principle of these devices are described first, and it is followed by the characterization features as gain, photon detection efficiency, linearity and noise effects.
\\

In the fourth chapter describes the simulation chain of the IDEA dual-readout calorimeter combining the Monte Carlo simulation of the calorimeter and the readout system simulation (i.e. a SiPM digitization software). Then a number of analyses are summarised to evaluate the performances and validate the full simulation.
\\

The last chapter presents the development and application of deep learning algorithms. Two different algorithms have been built with the aim to perform particle identification. The high spatial resolution of the calorimeter allows to fine reconstruct the 2D spatial distribution of released energy. Therefore, the neural networks are trained to analyse these distributions and distinguish neutral pions from photons.
\\