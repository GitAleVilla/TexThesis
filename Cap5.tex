\chapter{IDEA DR calorimeter full simulation}
As already said, the project described in chapter \ref{chap:Idea_project} is an on-going production and has to be supported by simulation.
With this goal, a dual-readout calorimeter full simulation has been developed allowing to generate data and monitor the whole process from the collision on the interaction point to the digitized signal produced by SiPMs.\\

The chapter presents a description of the simulation structure. The section \ref{sec:Sim_struc} describes in details the simulation dividing it in two main Monte Carlo processes:
\begin{itemize}
	\item the calorimeter simulation, coded in GEANT4;
	\item the SiPM response digitization ("pySIPM"), coded in Python.
\end{itemize}

Later, the performances obtained will be shown. The temporal behavior, the SiPM saturation effect and the energy resolution will be described in section \ref{sec:Sim_perf}.\\

The second half of the chapter treats of the possibility of simple particle identification using neural network structures.\\
In section \ref{sec:NN_waveform} neural networks working on digitized waveforms are described. The aim of these neural network is to correctly distinguish waveforms generated by electrons ($e^-$) or pions ($\pi^-$) in a range of energy from $20$ to $80$ GeV.\\
The last section (sec.\ref{sec:NN_img})  exposes another type of neural networks. These have the purpose of identify if signal are generated from photons ($\gamma$) or neutral pions ($\pi^0$) analyzing the spazial pattern of energy released in the calorimeter.

\section{Simulation structure} \label{sec:Sim_struc}
aaa

\subsection{Calorimeter simulation} \label{subsec:Sim_cal}
aaa

\subsection{SiPM response digitization} \label{subsec:Sim_SiPM}
aaa

\section{Simulation performances} \label{sec:Sim_perf}
aaa

\subsection{Different configurations} \label{subsec:SiPM_conf}
aaa

\subsection{Time studies} \label{subsec:Time}
aaa

\subsection{Saturation effect} \label{subsec:Sat_effect}
aaa

\subsubsection{Occupancy effect and Energy loss}
Studies of the occupancy effect are important preliminary studies that give knowledge about the information loss in the detection process.\\

\subsection{Digitization impact on energy resolution} \label{subsec:E_res}
aaa

\section{Neural Network: Particle ID on waveform} \label{sec:NN_waveform}
aaa

\subsection{Configuration}
aaa

\subsection{Performances}
aaa

\section{Neural Network: Particle ID on imaging} \label{sec:NN_img}
aaa

\subsection{Configuration}
aaa

\subsection{Performances}
aaa