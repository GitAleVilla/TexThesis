\chapter{Calorimetry and dual-readout}
Calorimetry is an important detection principle in particle physics.
Originally developed with astrophysical purpose for cosmic-ray studies, this method refers to the detection of particles and the measurement of their properties, using blocks of instrumented material.
It was developed and perfected for accelerator-based particle physics experimentation primarily in order to measure the energy of particles. 
In these blocks, particles are fully absorbed and their energy transformed into a measurable quantity.\\
The incident particle interact with the detector (through electromagnetic or strong processes) producing a shower of secondary particles with progressively degraded energy.
The energy deposited by the charged particles of the shower in the active material of the calorimeter, which can be detected in the form of charge or light, is used to measure the energy of the incident particle.
Typical processes suitable to detect this energy are: ionization of the medium, scintillation light and the Cherenkov light produced by relativistic particles.\\

Calorimeters can be divided into two categories depending on the type of shower they are optimized to detect: electromagnetic calorimeters, used mainly to measure electrons and photons through their electromagnetic interactions (e.g.
bremsstrahlung, pair production), and hadronic calorimeters, used to measure mainly hadrons through their strong and electromagnetic interactions.\\
Another classification can be made according to their construction technique defining sampling calorimeters and homogeneous calorimeters.\\
Homogeneous calorimeters are built of one type of material that performs both the main tasks: degrade the energy of the incident particles and provide the detectable signal.\\
Sampling calorimeters, instead, consist of alternating layers of an absorber, a dense material used to perform energy degradation, and an active medium that generate the signal.\\

%Calorimeters are attractive in high-energy particle physic field for various reasons:
%\begin{itemize}
%		\item In most cases the calorimeter energy resolution improves with energy as $1/\sqrt{E}$, where $E$ is the energy of the incident particle. Therefore calorimeters are very well suited to high-energy physics experiments.
%		\item Calorimeters are sensitive to all types of particles, charged and neutral (e.g., neutrons). Also neutrinos and their energy can be indirectly detected can even provide indirect detection of neutrinos and their energy through the measurement of the event missing energy.
%		\item They are versatile detectors. They can be used to determine the shower position and direction, to perform particle identification, to measure the arrival time of the particle, or even to provide fast signals useful in trigger purpose.
%		\item They are space and therefore cost effective. Because the shower length increases only logarithmically with energy, the detector thickness needs to increase only logarithmically with the energy of the particles.
%\end{itemize}

This chapter describes the physics behind both the electromagnetic and hadronic shower developments, provides a basic description of the energy response of these detectors and introduces the particular technique of the dual-readout, a modern concept of calorimeter that has the quality of overcome the non-compensating problem measuring both electromagnetic and hadronic showers through two different type of signals simultaneously (Cherenkov and scintillation light).\\

\section{Physics of shower development}
A particle interact and lose part or whole of its energy traversing matter. During this process the medium get excited and heated up, is excited in this process, or heated up. From this feature the term calorimetry, literally meaning "heat measurement", was introduced.\\
The groundwork for the calorimetry is the interaction processes between particle and matter. They are the manifestation of the electromagnetic, the strong and, more rarely, the weak forces and they strongly depend on the energy and the nature of the incident particle, in addition to medium features.\\
The processes and the consequent shower effects are the keys to deeply understand this topic.\\

\subsection{Electromagnetic showers} \label{subsec:em_shower}
Despite the complex mechanisms of particle-matter interaction, electromagnetic showers are produced via a small number of well unterstood QED processes. Charged particles (electrons and positrons) lose energy by ionization and by radiation, instead neutral ones (photons) are characterized by photoelectric effect, Compton scattering and pair production.

\subsubsection*{Electrons and positrons}

\subsubsection*{Photons}

\subsection{Hadronic showers} \label{subsec:had_shower}
aaa

\section{Energy response of calorimeters}
aaa

\subsection{Homogeneous calorimeters}
aaa

\subsection{Sampling calorimeters}
aaa

\subsection{Compensation}
aaa

\section{Dual-readout calorimetry}
aaa

\subsection{Working principles}
aaa

\subsection{Experiments}