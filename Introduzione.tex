\chapter{Introduction}
The discovery of the Higgs boson in 2012 at CERN has opened the door for several new studies in particle physics.
%Its relatively small mass allows to produce it in high luminosity electron-positron colliders.
The Higgs boson plays a crucial role in the biggest mysteries of modern particle physics and the precise measurement of its properties, together with the properties of the $W$ and $Z$ bosons, will provide deep information about the Standard Model (SM) and the physics Beyond it (BSM). At present, several projects are ongoing to design the next big-collider and its detectors with the primary task of measuring the Higgs properties.\\

In the first chapter, a short introduction to the main goals of modern particle physics is presented. The current design of possible future electroweak collider facilities is described, pointing out their capabilities to improve our knowledge of the field.
Finally, the IDEA detector concept is outlined, being one of the multi-purpose detector concepts at future electroweak colliders.\\

The second chapter provides a theoretical background on calorimetry, starting from the interaction between particles and matter and the particle showering mechanism. A description of modern calorimeters is also provided, distinguishing between electromagnetic and hadronic calorimeters. Eventually, the dual-readout compensation technique, cornerstone of the IDEA calorimeter, is introduced.\\

The third chapter is dedicated to the Silicon PhotoMultipliers. They are the photodetectors envisaged to be coupled to the IDEA dual-readout calorimeter. Their working principle is described, followed by their characteristic features such as gain, photon detection efficiency, linearity and noise sources.\\

The fourth chapter describes the simulation chain of the IDEA dual-readout calorimeter combining the Monte Carlo simulation of the calorimeter with the readout system simulation, emulating the SiPM response. Several analyses were performed to evaluate the performance and validate the full simulation model. They are described here for the first time.\\

The last chapter introduces the application of deep learning algorithms to the IDEA calorimeter simulated data. Two different algorithms have been used to perform the very challenging task of discriminating between neutral pions and photons. Indeed, the high spatial resolution of the IDEA calorimeter allows to reconstruct the two-dimensional pattern of particle showers with great accuracy, thus providing a powerful input for neural-network applications. Encouraging results were found and are extensively reported in the thesis.\\