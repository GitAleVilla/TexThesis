\chapter{Neural Network: Particle ID on imaging} \label{sec:NN_img}
Neural network and machine learning algorithm are powerful and versatile instruments that can improve the process of data analysis learning from datasets and making prediction with a certain degree of accuracy. Modern science, including physics, is taking more and more advantage of these techniques as the years go by.
The availability of big data is a key aspect in performing these studies and experimental particle physics is an area that can provide large datasets.\\

The chapter describes a possible application of deep neural networks on data generated through the IDEA DR Calorimeter full simulation with the aim to obtain simple particle identification analyzing the spatial deposited energy distribution.\\
An introduction to the computational techniques is provided, briefly describing the role of each component and the common structures that are used to perform the study.\\

All the data have to be prepared to be used in a neural network structure, the section \ref{sec:NN_data} shows which information has been used and how it has been processed to be analyzed by the deep learning algorithms.
After that the structures are presented in details showing the performances obtained in training and testing phases.\\
Eventually the study has been performed extending the energy range, the impact of this generalization conclude the chapter.\\
\newpage

\section{Project goal}
aaa

\section{Neural Networks introduction}
aaa

\subsection*{Dense layer}
\subsection*{Dropout layer}
\subsection*{Conv2D layer}
\subsection*{MaxPool2D layer}


\subsection*{VGGNet structure}
\subsection*{ResNet structure}


\section{Data setup}\label{sec:NN_data}
aaa

\section{Performances}
aaa

\section{Energy range extension}
aaa