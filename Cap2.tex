\chapter[Calorimetry and the DR method]{Calorimetry and the dual-readout method}
%Calorimetry is an important detection principle in particle physics.
Originally developed with astrophysical purpose for cosmic-ray studies, the art of calorimetry refers to the detection of particles and the measurement of their properties by total or partial absorption using blocks of instrumented material.
It was developed and perfected for accelerator-based particle physics experimentation primarily in order to measure the energy of particles. 
In calorimeters, particles are fully absorbed and their energy transformed into a measurable signal.\\
The incident particle interact with the detector (through electromagnetic or nuclear processes) producing a shower of secondary particles with progressively degraded energy.
The energy deposited by charged particles in the calorimeter active material is used to generate signals, typically in the form of charge or light.
Two typical processes exploited are the scintillation light emitted in response of ionization, or the Cherenkov light produced by relativistic particles.\\

Calorimeters can be divided into two categories depending on the type of shower they are optimized to detect: electromagnetic calorimeters, used mainly to measure electrons and photons energies through their electromagnetic interactions with the detector material, and hadronic calorimeters, used to measure hadrons energies through their strong and electromagnetic interactions.
Another classification can be made according to their structure dividing between sampling calorimeters and homogeneous calorimeters.\\
Homogeneous calorimeters are built of one type of material that performs two tasks: it degrades the energy of the incident particles and provides the detectable signal.
Sampling calorimeters, instead, consist of alternating layers of an absorber, a dense material used to absorb particle showers, and an active medium that generate the signal.\\

This chapter describes the physics behind both the electromagnetic and hadronic shower developments, provides a basic description of the energy response of these detectors and introduces the dual-readout calorimetry technique.\\

A more comprehensive descriptions of the field can be found in \cite{Wigmans_book, Wigmans_art_of_cal, Gianotti_article}.

\section{Physics of shower development}
The groundwork for the calorimetry is the interaction processes happening between particles and matter.
These processes depend on the absorbing medium, the particle type and energy.
While absorbing an energetic particle, a cascade of subsequent particles is formed known as a particle shower.
The processes involved and the calorimeter response to the showering particles are the keys to deeply understand this topic.\\

\subsection{Electromagnetic showers} \label{subsec:em_shower}
Electromagnetic showers are governed by a small number of well understood QED processes. Charged particles (electrons and positrons) lose energy by ionization and by radiation, instead neutral ones (photons) interact with matter by photoelectric effect, Compton scattering and pair production.\\

Therefore, electrons and positrons ionize the medium under the condition of having an energy at least sufficient to release the atomic electrons from the Coulomb fields generated by the atomic nuclei (few eV).
The amount of energy released (in unit of path length) is predictable through the semi-empirical Bethe-Block formula restricted to electrons (and positrons) \cite{Leo}:
\begin{equation}
    -\frac{dE}{dx} = 2\pi N_a r_e^2 m_e c^2 \rho \frac{Z}{A}\frac{1}{\beta^2}\left[ \ln{\frac{\tau^2(\tau + 2)}{2(I/m_ec^2)^2}} -F(\tau) -\delta -2\frac{C}{Z}\right].
\end{equation}
The stopping power (i.e. $dE/dx$) decreases as the particle energy increases ($\propto \beta^2$). Hence the ionization process is the greatest source of energy loss for particles with small energy.\\
The radiative energy loss process known as \textit{bremsstrahlung} is the dominant source of energy loss by electrons and positrons at energies above $10-100$ MeV, depending on the absorber material. Relativistic electrons and positrons radiate photons as a result of the interaction between Coulomb and the atomic electric fields. The energy spectrum of these photons falls off as $1/E$ ranging till the primary particle energy, but in general most of the photons carry a small part of it.
The process produces (usually small) changes in electron (or positron) direction, contributing to the Coulomb or multiple scattering.\\
At a fixed energy the relative importance of ionization and radiation losses depends on the medium and in particular on its electron density.
This density is in first approximation proportional to the number of protons in the nuclei ($Z$).
The critical energy, i.e. the energy value at which the two processes have equal impact, is roughly inversely proportional to the $Z$ value of the material:
\begin{equation}
    \varepsilon_c = \frac{160\text{ MeV}}{Z + 1.24}.
\end{equation}
An example of energy loss in copper by electron is sketched in Figure \ref{fig:Cu_rad_ion}, where the ionization and radiation contribution are plotted.\\

\begin{figure}
	\centering
	\includegraphics[width=0.8\textwidth]{IMG/Cap2/Cu_rad_ion}
	\caption{Energy losses through ionization and bremsstrahlung by  electrons in copper \cite{PDG_98}.}
	\label{fig:Cu_rad_ion}
\end{figure}

On the other hand, the interaction between photons and matter is mainly affected by three different processes: the photoelectric effect, the Compton scattering and the electron–positron pair production.\\
The photoelectric effect is the process that most likely occurs at low energies. It is characterized by an atom absorbing the photon and emitting an electron. The photoelectric cross section strongly depends on the available number of  electrons,  and  thus  on  the $Z$ value  of  the  absorber  material. In particular it scales with $Z^n$, with the power $n$ between $4$ and $5$. The photoelectric cross section rapidly decrease with greater energies, varying as $E^{-3}$, and the process rapidly loses its impact as the energy increases.

The Compton process is a scattering process where an impinging  photon interact with an atomic electron transferring enough momentum and energy to the struck electron to escape from the atomic Coulomb field. Kinematic variables such as energy transfer and scattering angles can be easily obtained applying the laws of energy and momentum conservation.

Photons in the MeV energy range are absorbed by photoelectric effect only after a sequence of Compton scattering processes, in which the photon energy is reduced step by step in each collision until it is low enough to favour the photoelectric occurrence. In each step, the photon energy loss is:
\begin{equation}
    T = E_{\gamma}\frac{\xi(1 - \cos{\theta})}{1 + \xi(1 - \cos{\theta})}
\end{equation}
where $\xi = E_{\gamma}/m_ec^2$.
The Compton scattering cross section is much less dependent on the $Z$ value than the photoelectric one. It is almost proportional to the number of target electrons in the nuclei. Also in this process the cross section decreases with increasing photon energy, but only with the first power of $E$. Therefore Compton scattering has more impact than photoelectric absorption above a certain threshold energy. 

The pair production process, differently from the previous ones, has a lower threshold under which the effect can not occur. This threshold is twice of the electron rest mass ($2\times 511$ MeV). Above the threshold, a photon can produce an electron-positron pair.
The cross section for pair production rises with the energy reaching an asymptotic value at energies higher then $1$ GeV. For this reason, at high energies, pair production is the most likely process to occur. Meanwhile the dependence from the medium goes, in first approximation, as $Z^2$.\\

Comparing the cross section of the three processes and their dependence to the photon energy it is clear that the photoelectric effect dominates at lower energies, While at the intermediate values the Compton scattering gives the greatest contribution.
At higher energies almost every photon is converted in charged particles through pair production. An example of these contributions is shown in Figure \ref{fig:ph_cross_E}.\\
Knowing the dependence of the cross sections with respect to the $Z$ of the material, ranges of energies where each process dominates can be found and parametrized with the $Z$ value. A representation is sketched in Figure \ref{fig:ph_cross_Z}.

\begin{figure}
	\centering
	\subfloat[][Total cross section of photon in Carbon. Different processes contribution are also separated. \label{fig:ph_cross_E}]{\includegraphics[height=.18\textheight]{IMG/Cap2/ph_cross_E.png}} \quad
	\subfloat[][Energy ranges where different processes dominate with respect to the medium $Z$ value.\label{fig:ph_cross_Z}]{\includegraphics[height=.18\textheight]{IMG/Cap2/ph_cross_Z.png}}
	\caption{Images from \cite{Leo}.}
	%\label{fig:sigma_su_e}
\end{figure}

\subsubsection*{Electromagnetic shower principle}
%Minimal showers may also develop at very low energy of primary particle. Starting for example from a photon of tens of $MeV$, it can eventually produce a electron-positron pair in the calorimeter. The  charged  particles lose their energy in the matter through ionization. When the  positron loses all the kinetic energy, it annihilates with an electron producing two $511$ keV $\gamma$s. These photons are absorbed through the photoelectric effect after a sequence of Compton scattering. During the process, the energy of the primary particle is released to the material by charged particle in ionization processes.\\
At energy values of $1$ GeV and higher, electrons, protons and photons initiate electromagnetic showers in the materials in which they penetrate. At these energies charged particles lose their energy mostly by brehemstralung, the majority of these photons are very soft, and interact with Compton scattering until their absorption through photoelectric effect. Meanwhile the photons with energy more than $5–10$ MeV produce $e^+-e^-$ pairs, which eventually radiate more $\gamma$'s. The process continues till the photon energy is enough to continue the particle multiplication. The shower maximum is defined as the point at which the number of shower particles produced in this particle multiplication process reaches its maximum. The depth inside the absorber associated to the shower maximum increases logarithmically with the energy of the primary particle (see Figure \ref{fig:shower_max}). The longitudinal shower development is described by the radiation length ($X_0$), it is defined as the distance at which the electron (or positron) loses on average $63\%$ $(1-e^{-1})$ of its energy by radiation. Expressing the shower containment in term of $X_0$ is useful to mitigate the dependence of the shower containment on the absorber material.

\begin{figure}
	\centering
	\includegraphics[width=0.7\textwidth]{IMG/Cap2/shower_max.png}
	\caption{The energy deposit as a function of depth, for $1$, $10$, $100$ and $1000$ GeV electron showers developing in a block of copper \cite{Leo}.}
	\label{fig:shower_max}
\end{figure}

Another quantity useful to describe the spatial shower development, in particular the transverse one, is the Molière radius. It is defined in terms of the radiation length and the critical energy:
\begin{equation}
    \rho_M = E_s \frac{X_0}{\varepsilon_c}
\end{equation}
where $E_s$ is defined as $m_c^2\sqrt{4\pi/\alpha} \simeq 21.2$ MeV. This quantity is almost material-independent and, on average, a cylindrical volume with this radius around the shower axis contains $90\%$ of the shower energy. The lateral spread is mainly caused by two effects: at high energy, electrons and positrons are moved away from the shower axis because of the deviation occurring in Compton scattering; photons and electrons are also produced in isotropic processes moving them away from the axis (spread more important in lower energy particles). Also brehemstralung process produces photons with a certain angle, contributing to the shower lateral dimension. Figure \ref{fig:shower_moliere} shows the electromagnetic shower radial profiles at different showering stages.\\

\begin{figure}
	\centering
	\includegraphics[width=0.6\textwidth]{IMG/Cap2/shower_moliere.png}
	\caption{The radial distributions of the energy deposited by $10$ GeV electron showers in copper, at various depths \cite{Leo}.}
	\label{fig:shower_moliere}
\end{figure}

The lateral and longitudinal shower development generated by charged particles and by neutral ones are basically identical except for the initial stages. Electrons start radiating as soon as they enter the calorimeter, instead photons must convert before releasing any energy. Once they start producing electrons and positrons, they can release even more energy than electron induced showers. This behaviour is shown in Figure  \ref{fig:em_start}, where the distribution of the energy fraction deposited in the first $5\ X_0$ by $10$ GeV electrons and photons in lead is plotted.

\begin{figure}
	\centering
	\includegraphics[width=0.7\textwidth]{IMG/Cap2/em_start.png}
	\caption{Distribution of the energy fraction deposited in the first $5\ X_0$ by  $10$ GeV electrons and photons showering in lead. Image from \cite{Wigmans_e_gamma}.}
	\label{fig:em_start}
\end{figure}

\subsection{Hadronic showers} \label{subsec:had_shower}
Introducing the hadronic showers a new degree of complexity arises, indeed, showers produced by hadrons are also affected by strong interactions. This interaction is responsible for:
\begin{itemize}
    \item The production of hadronic secondary particles, most of which ($\simeq 90\%$) are pions. Neutral pions mainly decay in two photons, which develop electromagnetic showers.
    \item The occurrence of nuclear reactions where atomic nuclei release neutrons and protons. The fraction of the shower energy needed to unbind the nucleons does not contribute to any signal formation, and therefore is known as invisible energy.
\end{itemize}

Among the particles in an hadronic shower, neutral pions play an important role. Via their $\pi^0\rightarrow \gamma\gamma$ decay, they start electromagnetic showers over-imposed to the hadronic part. The fraction of energy carried by the induced em showers is called electromagnetic fraction ($f_{em}$).
On average, $f_{em}$ increases with the primary particle energy, since $\pi$'s may also be produced by secondary and higher-order particles: the higher the energy, the more the generations of shower particles, the larger the average em fraction. The  average  electromagnetic fraction has been evaluated to increase with the energy following the power law:
\begin{equation}
    f_{em} = 1 - \left(\frac{E}{E_0}\right)^{k-1}
\end{equation}
where $K\simeq 0.82$ and $E_0$ is a material dependent value of the GeV order. The behaviour in lead and copper is shown in Figure \ref{fig:f_em}.\\
Another contribution to the complexity is given by the variability of this em fraction event by event and its energy dependence.\\

\begin{figure}
	\centering
	\includegraphics[width=0.7\textwidth]{IMG/Cap2/f_em.png}
	\caption{Comparison between the experimental results on the em fraction of pion-induced showers in copper-based and lead-based calorimeters. Image from \cite{fem_par}.}
	\label{fig:f_em}
\end{figure}

Another important difference between em and hadronic showers is the greater spatial profiles of released energy. Analyzing for example several pions showers, peaks of signals are produced at a depth near the one where a $\pi^0$ is generated. However, the neutral pion production can occur in the second or third generation of the shower development releasing energy at different depth. An example of $4$ different showers from $270$ GeV pions is illustrated in Figure \ref{fig:had_start}.\\
The depth of the calorimeter required to contain hadronic showers increases logarithmically with energy, as already seen for em showers, but the large longitudinal fluctuations in the showering starting point make the leakage effect non negligible also in configurations that would contain, on average, $99\%$ of the shower.\\
Laterally, an hadronic shower is more contained if the primary particle energy increases. This is due to the fact that the electromagnetic shower fraction increases with energy and the em showers produced tend to develop laterally closer to the shower axis.
 
 \begin{figure}
	\centering
	\includegraphics[width=0.7\textwidth]{IMG/Cap2/had_start.png}
	\caption{Longitudinal  profiles  for  $4$  different  showers  induced  by  $270$ GeV pions in a lead/iron/plastic-scintillator calorimeter \cite{Wigmans_art_of_cal}.}
	\label{fig:had_start}
\end{figure}

\section{The calorimeter response}
The calorimeter \textit{response} is defined as the average calorimeter signal per unit of deposited energy. The response is usually expressed in terms of number of photoelectrons per GeV or charge (pC) per MeV.
A calorimeter with a constant response is said to be \textit{linear}.
In the following we compare the response of electromagnetic and hadronic calorimeters.\\

\subsection*{Electromagnetic calorimeters}
Electromagnetic calorimeters are detector optimized to absorb em showers. In these showers, all the energy carried by the primary particle is released by a very large number of particles through processes that may generate signals (excitation or ionization in the absorbing medium). The number of particles is on average proportional to the primary energy. The direct consequence is that em calorimeters are in general linear detectors.\\
A non-linear response is usually an indication of instrumental problems. The most common ones are:
\begin{itemize}
    \item leakage effects;
    \item saturation effects of the active components, originated by high localized energy depositions;
    \item recombination of electrons and ions, making the energy carried by the electron undetected.
\end{itemize}

At the same time, the small lateral and longitudinal em shower development allows to design relatively small calorimeters, often used as electromagnetic components in more complex systems.\\

The small dimension also allows to build em calorimeters operating at colliders both of the homogeneous and sampling type.

The main advantage of homogeneous detectors is their excellent energy resolution, achievable because the whole energy of the primary particle is released in the active medium. At the same time, if a well position measurement and particle identification are required, homogeneous calorimeters are not the best choice because of the larger shower dimensions task. 
Homogeneous calorimeter can be classified in four groups:
\begin{itemize}
    \item semiconductor calorimeters;
    \item Cherenkov calorimeters;
    \item scintillator calorimeters;
    \item noble-liquid calorimeters.
\end{itemize}

On the other hand, sampling calorimeters have in general a worse energy resolution due to the presence of an absorber material (copper, iron, lead or uranium) that does not contribute to the signal. For electromagnetic sampling calorimeters, typical resolution values are in the range of $5-20 \% / \sqrt{E(\text{GeV})}$. However, they are relatively simple to segment longitudinally and laterally due to their sampling structure often helping to achieve a better spatial resolution.

\subsection*{Hadronic calorimeters}
Hadronic calorimeters are detector optimized to absorb hadronic showers. At collider experiments, to limit the detector dimensions they are always of the sampling type. The calorimeter response is a mixture between the response to the em ($e$) and non-em ($h$) components. The ratio $e/h$ classifies calorimeters in three categories: \textit{compensating}, if $e/h=1$; \textit{undercompensating}, if $e/h>1$; \textit{overcompensating}, if $e/h<1$. Hence, the total calorimeter response (to hadrons) is a combination of the two:
\begin{equation}
    \pi = f_{em} \cdot e + (1- f_{em}) \cdot h.
\end{equation}
Since the average $f_{em}$ value, as already seen, increases with the energy, the response of a non-compensating calorimeter ($e/h \neq  1$) is not constant and non-compensating calorimeters are intrinsically non-linear detectors.\\
The $e/h$ value cannot be directly measured. However, it is usually derived from the $e/\pi$ ratios, measured at various energies. The relationship between $e/\pi$ and $e/h$ is:
\begin{equation}
    \frac{e}{\pi} = \frac{e/h}{1 - f_{em}(1-e/h)},
\end{equation}
where $f_{em}$ is the energy dependent average em fraction. A calorimeter for hadron detection does not present a linear response with respect to the primary particle due to the energy dependence of the em fraction. The relation is represented in Figure \ref{fig:compensation} for compensating and non-compensating calorimeters.\\

 \begin{figure}
	\centering
	\includegraphics[width=0.65\textwidth]{IMG/Cap2/compensation.png}
	\caption{Relation between the calorimeter response ratio to em and non-em energy deposits, $e/h$, and the measured $e/\pi$ signal ratios \cite{Wigmans_art_of_cal}.}
	\label{fig:compensation}
\end{figure}

The basic reason for different response for em and hadronic showers lies in the fact that in the absorption of hadronic showers, a significant fraction of the energy is \textit{invisible} not contributing to the calorimeter signal. The main source of this energy is the energy used to release nucleons from nuclei, the invisible energy.\\
To design a linear calorimeter various compensation methods have been developed. There are two main approaches to obtain $e/h = 1$: one is reducing the electromagnetic response and the other one is to increase the non-em response.\\
The most effective way to reduce the em response of a sampling calorimeter is to use an high-$Z$ material as absorber. The concept is based on the fact that low energy photons release their energy via the photoelectric effect that is highly $Z$ dependent ($Z^5$). For this reason, using high-$Z$ absorber materials causes an energy deposition in the absorber making their energy less likely to be detected in the active elements.\\
The other strategy is known as compensation by neutrons signal boosting or by Signal Amplification through Neutron Detection (SAND). It consists in increasing the non-em response taking advantage of the kinetic energy transported by neutrons. The good correlation between the invisible energy and the kinetic energy of neutrons gives the possibility of an event-by-event correction boosting the signals generated by the neutrons. The most likely process that occur at low energies for a neutron is elastic scattering on a nucleus target. The fraction of energy transferred in this process is on average $f_{\text{elastic}} = 2A / (A +1)^2$ (where $A$ is the mass number of the target). Considering that hydrogen maximize this fraction it is the element that must be present in the active medium to produce the non-em response increment. With a proper fine tuning of the sampling fraction, a $e/h$ ratio of $1$ can be reached.
In short, compensation through SAND can be achieved in sampling calorimeters with active materials containing hydrogen and a precisely tuned sampling fraction.\\
In the IDEA dual-readout calorimeter, a different compensation strategy is adopted the dual-readout compensation that will be described in Section \ref{sec:DRComp}.

\subsection{Fluctuations}
Event by event fluctuations in the calorimeter response determine the detector energy resolution. The energy resolution is quantified as the reconstructed energy distribution standard deviation divided by the mean value ($\sigma/E$).\\
In em calorimeters, four are the main sources of response fluctuation:
\begin{itemize}
    \item sampling fluctuations;
    \item signal quantum fluctuations;
    \item shower leakage fluctuations;
    \item instrumental fluctuations.
\end{itemize}
Sampling effects are affected both by the sampling fraction (i.e. the ratio of the active material on the total) and the sampling frequency (the thickness of the layers). The sampling fraction is  defined as:
\begin{equation}
    f_{samp} = \frac{E_{active}}{E_{passive}+E_{active}}
\end{equation}
where $E_{active}$ and $E_{passive}$ are the energies deposited in the active and passive part by an incident minimum ionizing particle (mip).
This type of fluctuation is dominated by the Poisson statistics, hence it contributes with a term proportional to $1/\sqrt{E}$ to the energy resolution.
In em calorimeter with non-gaseous active material the sampling contribution follows the empirical law:
\begin{equation}
    \frac{\sigma}{E} = \frac{2.7\% \sqrt{d/f_{samp}}}{\sqrt{E}}
\end{equation}
where $d$ is the thickness of the active material layer measured in mm.\\
Fluctuation effects due, for example, to photon statistics in light emitting active materials are grouped under the label of signal quantum fluctuations. Also these fluctuations follow the rules of Poisson statistics, with the constrain of uncorrelated separate contributions. Another contribution that scales with $E^{-1/2}$ is added to the relative energy resolution expression.\\
In leakage fluctuations there are three possibilities: longitudinal, lateral and albedo. These fluctuation are highly dependent to the geometry and the material of the calorimeter. For this reason their contribution to the resolution does not have a precise energy dependent form. The typical solution to numerically evaluate their effect is via Monte Carlo simulation of the detector.\\
Finally, instrumental fluctuations are the ones provided, for example, by electronic noise or geometrical inhomogeneities. The electronic noise contribution is usually described with a term that scales with $1/E$ because it is largely independent of the shower energy. Meanwhile fluctuations induced by structural inhomogeneities depend on the shower position and might be energy dependent.\\
Typically, these contribution are uncorrelated and have to be added in quadrature to obtain the total resolution. As seen, different contribution with different energy dependence have to be considered. The consequence is that different effects dominate the energy resolution in different energy ranges. These behaviour have to be considered in the design of the calorimeter in order to optimize the resources.\\

All the sources described affect also hadronic calorimeters. For example, sampling fluctuation has a higher impact in hadronic showers due mainly to the lower average number of particle that release energy in the hadronic ones. 
At the same time, in hadronic detectors, resolution is affected also by other fluctuation effects.\\
%The introduction of quantities such as $f_{em}$ in hadronic calorimeters produces new sources of fluctuations.
In non-compensating calorimeters, event-by-event fluctuation in em component affect the detector response. It contributes to the relative energy resolution through a term scaling as $c E^{-0.28}$.
\begin{equation}
    \frac{\sigma}{E} = \frac{a_1}{\sqrt{E}} \oplus c E^{-0.28}
\end{equation}
Up to $400$ GeV, this law runs almost parallel to an energy resolution in which only the stochastic term is included ($\sigma/E = a/\sqrt{E}$). For this reason, the resolution of non-compensating calorimeter is often described as:
\begin{equation}
    \frac{\sigma}{E} = \frac{a_2}{\sqrt{E}} + b.
\end{equation}
A useful solution that remove these fluctuations is to design a compensating calorimeter, most often improving the energy resolution.\\
%Finally, the calorimeter resolution is also affected by invisible energy fluctuations. To evaluate the impact of these fluctuations the correlation between $f_{em}$ and $f_{inv}$ has to be known, and in particular if they are correlated or not. If they are uncorrelated the contribution to the resolution has to be added in quadrature. Moreover, in compensating calorimeter, the contribution of invisible energy fluctuations are different depending on the techniques used to achieve compensation.

\section{Dual-readout compensation}\label{sec:DRComp}
All the benefits obtained by compensating calorimeters are achieved by sampling em and non-em components in hadronic showers with the same response ($e=h$). Dual-readout calorimetry is a method that measures the electromagnetic fraction on an event-by-event basis and applies a correction, reaching $e/h = 1$ value \cite{DR_Wig}. 

\subsection{Working principles}
This technique is based on the concept that the em component is mostly brought by relativistic electrons and positrons, while the majority of non-em one is carried by non-relativistic particles. Hence, collecting the Cherenkov signal produced by an hadronic shower is almost equivalent to sampling the em component, thus measuring the electromagnetic fraction. A second simultaneous signal, typically from scintillators, is recorded to provide information on the total event ionizing energy.\\ %Thanks to this second signal, the non-em component can be evaluated starting from the $f_{em}$ already obtained.\\

Let's consider scintillation ($S$) and Cherenkov ($C$) signals produced by an hadronic shower developing in a dual-readout calorimeter. The mean values of these signals are calibrated with an electron beam of known energy $E$ so that, for em showers, $\expval{S} = \expval{C} = E$. The hadronic signals can be written (for each event) as:
\begin{align}
    S &= E \left[f_{em} + (h/e)_S (1-f_{em}) \right], \label{eq:S} \\
    C &= E \left[f_{em} + (h/e)_C (1-f_{em}) \right], \label{eq:C}
\end{align}
where $h/e$ quantifies the two different degree of non-compensation for the two signals. Considering that $h/e$ are measurable quantities and are assumed to be constants, the ratio $C/S$ is also energy independent:
\begin{equation}
    \frac{S}{C} = \frac{f_{em} + (h/e)_S (1-f_{em})}{f_{em} + (h/e)_C (1-f_{em})}. \label{eq:em_frac}
\end{equation}
This expression can straightforwardly give an evaluation of the em fraction:
\begin{equation}
    f_{em} = \frac{(h/e)_C-(C/S)(h/e)_S}{(C/S)\left[1-(h/e)_S\right]-\left[1-(h/e)_C\right]},
\end{equation}
once the $h/e$ values are known.\\
%allowing the rewriting of \ref{eq:S} and \ref{eq:C} as
%\begin{align}
%    S/E &= (h/e)_S + f_{em}\left[1-(h/e)_S\right], \label{eq:S2} \\
%    C/E &= (h/e)_C + f_{em}\left[1-(h/e)_C\right]. \label{eq:C2}
%\end{align}

An example of a $C/E$ vs $S/E$ scatter plot is shown using signals from RD52 DR calorimeter for electrons, pions and protons showers. As expected, data generated from electrons are always around the point $[1,1]$ (as $f_{em} = 1$). On the other hand an hypothetical event with only the non-em component would be at $[(h/e)_S,(h/e)_C]$. Signals from pions and protons showers are clustered along a straight line linking these two points.
%From this geometric construction it is clear that the slope of this line (the $\theta$ angle) is energy independent because it is determined only by the two $e/h$ values.
Thanks to the fact that $\theta$ is energy and particle independent, it is possible to define the parameter:
\begin{equation}
    \chi = \frac{1-(h/e)_S}{1-(h/e)_C} = \cot{\theta},
\end{equation}
this parameter can be estimated with test beam data.
Finally, event-by-event, the energy of each hadron shower, corrected for the $f_{em}$ value, can be reconstructed using the $S$ and $C$ signals as:
\begin{equation}
    E = \frac{S - \chi C}{1 - \chi}.
\end{equation}

\begin{figure}
	\centering
	\includegraphics[width=0.65\textwidth]{IMG/Cap2/theta_DR.png}
	\caption{Relation between the calorimeter response ratio to em and non-em energy deposits, $e/h$, and the measured $e/\pi$ signal ratios \cite{DR_Theta}.}
	\label{fig:theta_DR}
\end{figure}

The ideal DR calorimeter would able to identify the Cherenkov signal as the exact electromagnetic component ($h/e = 0$) being a direct measurement of $f_{em}$. The worst DR calorimeter instead would have $(h/e)_C = (h/e)_S$, in this case the two signals would sample the two components (em and non-em) with the same response giving no information about the $f_{em}$. Therefore, the best DR calorimeter is the one with the lower $\chi$ parameter corresponding to the lower $(h/e)_C$ and the higher $(h/e)_S$ values possible. 

\subsection{Dual-readout sampling calorimeters}
At present a few dual-readout calorimeter have been built and tested, and more project are under study.\\
The first practical demonstration of the dual-readout method was achieved by a R\&D study for the Advanced Cosmic Composition Experiment at the Space Station (ACCESS) \cite{ACCESS}. The prototype had a depth of $1.4\ \lambda_{int}$ and was equipped with both plastic-scintillator ($S$) and quartz ($Q$) optical fibres to measure and collect the scintillation and Cherenkov light, respectively. With this detector, the $Q/S$ signals ratio was found to represent a good event-by-event measurement of the shower energy fraction carried by $\pi^0$.\\

Later, a $10\ \lambda_{int}$ deep calorimeter known as Dual-REAdout Module (DREAM) was built and tested \cite{DREAM1, DREAM2}. 
The detector is composed by $5580$ basic element represented in Figure \ref{fig:DREAM}. These are $200$ cm long, hollow, extruded copper rod of $4\times 4$ mm$^2$ cross section in which $3$ scintillating and $4$ Cherenkov fibres are inserted.
\begin{figure}
	\centering
	\includegraphics[width=0.65\textwidth]{IMG/Cap2/DREAM.png}
	\caption{The DREAM calorimeter layout \cite{DREAM}.}
	\label{fig:DREAM}
\end{figure}

Figure \ref{fig:SC_Dream_sig} shows the energy distributions for $100$ GeV $\pi^-$ by the two fibre types, after a calibration at the electromagnetic scale. The distributions are peaked at considerably lower values than the one corresponding to electrons.
Populating a scatter plot with Cherenkov and scintillation signals for each event, the correlation between them is evident as shown in Figure \ref{fig:theta_DREAM}.
The two $h/e$ ratios for Cherenkov and scintillator DREAM structures were measured to be $0.21$ and $0.77$, respectively. The expression of the em fraction \ref{eq:em_frac} becomes:
\begin{equation}
    f_{em} = \frac{0.21-0.77(C/S)}{(C/S)\left[1-0.77\right]-\left[1-0.21\right]}.
\end{equation}

Being able to evaluate $f_{em}$ event-by-event open the possibility to correct the signals for the effects of non-compensation.
In Figure \ref{fig:fem_subsamp} the overall Cherenkov signal distribution for $100$ GeV $\pi^-$ (a) is shown, where the asymmetry from $f_{em}$ fluctuations is evident. In Figure \ref{fig:fem_subsamp}(b) the events are grouped in subsamples selected on the basis of their $f_{em}$ value.
Each one of these subsamples reproduce a certain region of the overall signal distribution, with the average value that increases with $f_{em}$ as expected. This representation gives an idea of the effectiveness of the method.\\
In this process, the energy resolution improved, the signal distribution became much more Gaussian and, most importantly, the hadronic energy was, on average, correctly reproduced, i.e. liearity greatly improved.\\

\begin{figure}
	\centering
	\includegraphics[width=0.65\textwidth]{IMG/Cap2/SC_Dream_sig.png}
	\caption{Signal distributions for $100$ GeV $\pi^-$ recorded by the scintillating (a) and the Cherenkov (b) fibers. \cite{Wigmans_art_of_cal}.}
	\label{fig:SC_Dream_sig}
\end{figure}

\begin{figure}
	\centering
	\includegraphics[width=0.65\textwidth]{IMG/Cap2/theta_DREAM.png}
	\caption{The $C-S$ scatter plot showing the correlation between the two signals. The signals are expressed in the same units used to calibrate the calorimeter (em GeV) \cite{DREAM2}.}
	\label{fig:theta_DREAM}
\end{figure}

\begin{figure}
	\centering
	\includegraphics[width=0.65\textwidth]{IMG/Cap2/fem_subsamp.png}
	\caption{Cherenkov signal distribution for $100$ GeV $\pi^-$ (a) and distributions for subsamples of events selected on the basis of the measured $f_{em}$ value, using the $Q/S$ method (b). Image from \cite{Wigmans_art_of_cal}.}
	\label{fig:fem_subsamp}
\end{figure}

The DREAM results, together with other studies, confirmed the feasibility of the dual-readout compensation method, leading to the IDEA dual-readout calorimetry project.